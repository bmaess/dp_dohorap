1.2 Functional anatomy of syntax processing

Region of interest: pSTS
pSTG is involved in complex syntax
Adults:
"Neuroimaging studies have tried to characterise the neural substrate for representing human action. Many of these studies have followed a different tradition in psychophysics and developmental psychology of investigating the perception of “biological motion” – that is, the characteristic articulated motion of chordate animal bodies (e.g. Vaina, Solomon, Chowdhury, Sinha, & Belliveau, 2001; Grossman & Blake, 2002; Beauchamp, Lee, Haxby, & Martin, 2002; Pelphrey, Mitchell, McKeown, Goldstein, Allison, & McCarthy, 2003) or body and face parts (e.g. Hoffman & Haxby, 2000; Hooker et al. 2003; Kilts et al. 2003; Pelphrey et al., 2003). Biological motion can also be perceived from the relative motion of just a few dots (“point-light walkers”, Johansson, 1973); if the dots are spatially or temporally rearranged, the percept is destroyed. These neuroimaging studies suggest that one brain region, the posterior superior temporal sulcus, is particularly involved in the representation of biological motion.
Two sets of recent neuroimaging data suggest that the role of the posterior superior temporal sulcus (pSTS) may extend beyond a response to biological motion, to more abstract representations of intentional action. First, Castelli, Happe, Frith, & Frith, (2000) and Schultz, Grelotti, Klin, Kleinman, Van der Gaag, Marois, & Skudlarski, (2003) reported that a region of the pSTS showed a significantly higher response to animations of moving geometric shapes that depicted complex social interactions than to animations depicting inanimate motion. Second, using movies of human actors engaged in structured goal-directed actions (e.g. cleaning the kitchen), Zacks, Braver, Sheridan, Donaldson, Snyder, Ollinger, Buckner, & Raichle, (2001) found that activity in the pSTS was enhanced when the agent switched from one action to another, suggesting that this region encodes the goal-structure of actions. Both of these results are consistent with a role for a region of pSTS cortex in representing intentional action, and not just biological motion." (doi:10.1016/j.neuropsychologia.2004.04.015)


"hierarchy construction might be supported by left inferior frontal (particularly BA44/45) and posterior superior temporal regions: (Ben-Shachar et al., 2003, 2004; Caplan et al., 2002; Just et al., 1996; Röder et al., 2002; Stromswold et al., 1996)"

"the enhanced inferior frontal (BA 44/45) activation observed for these sentence types is engendered by highly specialized aspects of syntactic processing, namely by syntactic transformations (Ben-Shachar et al., 2003, 2004; Grodzinsky, 2000)."

 "More complex syntactic processing, 
 including the identification of hierarchical structures within a sentence, takes 
 place later at around 300ms to 500ms in the already mentioned pSTG/STS 
 (Cooke et al., 2002; Vandenberghe et al., 2002; Humphries et al., 2005; 
 Bornkessel and Schlesewsky, 2006; Kinno et al., 2008; Friederici et al., 2009; 
 Snijders et al., 2009; Santi and Grodzinsky, 2010; Newman et al., 2010). How-
 ever, this is not achieved by the pSTG/STS alone but, as Friederici argues, 
 critically involves the pars opercularis of the inferior frontal gyrus (IFGoper) in 
 Broca’s  area  (Friederici et al., 2006; Makuuchi et al., 2009; Newman et al., 
 2010; Wilson et al., 2010)." (Skeide 2013)

 ELAN

Children:

"Children
 performed   best   on   subject   relatives,   followed   by   object   relatives,   indirect
 object relatives, oblique relatives and, finally, genitive relatives. Diessel and
 Tomasello argued that children’s superior performance on subject relatives
 reflected   a   processing   effect   whereby   children   prefer   to   pursue   subject-
 extracted   interpretations   because   they   have   a   preference   to   build   simple
 structures,   which   is   based   upon   their   considerable   experience   with   simple
 nonembedded sentences. This is similar to early arguments made by Bever
 (1970, see also Bates & MacWhinney, 1982; Slobin & Bever, 1982; Townsend
 & Bever, 2001), who argued that children use a canonical sentence schema
 (NVN)   to   interpret   sentences" (Kidd, Brandt, Lieven 2007)

Development of simple sentence processing is complete at age 3 (Friederici 2005)
Kids often don't have the AF tract yet until early adulthood
Previous findings: kids rely more on the ventral pathway
Ventral processing involves BA45 (no condition effect in adults)
Processing is more vulnerable to bias (semantic crosstalk)