\section{Sensor-space activity}

We computed average event-related fields (ERF) for each subject and sensor region.
Activity from these ERF was selected with two different types of time windows.

A positive effect indicates that activity evoked by object-relative clauses was more positive than activity evoked by subject-relative clauses. In the case of gradiometers, "'more positive"' means a higher regional RMS. With localized data, "'more positive"' means a higher z-score from the sLORETA source reconstruction.


\subsection{Interval analysis}
For this analysis, sensor activity from separate regions and hemispheres was compared blindly between 0 and 2200ms after onset in 200ms intervals.

After FDR-correction for 10 comparisons, no sensor region showed any non-spurious effect in either group ($p > 3\%$).

\subsection{Cluster analysis}
For this analysis, activity was compared between conditions using a temporal cluster analysis.

For children, significant differences were observed in the following sensor groups:
Right parietal magnetometers showed a negative effect between 159ms and 374ms ($p = 1.4\%$).
Left frontal magnetometers showed a negative effect between 375ms and 666ms ($p = 1.8\%$).
Left temporal magnetometers showed a negative effect between 349ms and 627ms ($p = 0.8\%$).
Right temporal gradiometers showed a positive effect between 1384ms and 1663ms ($p = 3.8\%$).

\begin{figure}[!h]
\begin{center}
\includegraphics[width=0.49\textwidth]{pics/children_Left-frontal-magnetometer.png}
\includegraphics[width=0.49\textwidth]{pics/children_Right-frontal-magnetometer.png}
\includegraphics[width=0.49\textwidth]{pics/children_Left-frontal-gradiometer.png}
\includegraphics[width=0.49\textwidth]{pics/children_Right-frontal-gradiometer.png}
\includegraphics[width=0.49\textwidth]{pics/kids_Left-frontal_mag.png}
\includegraphics[width=0.49\textwidth]{pics/kids_Right-frontal_mag.png}
\includegraphics[width=0.49\textwidth]{pics/kids_Left-frontal_grad.png}
\includegraphics[width=0.49\textwidth]{pics/kids_Right-frontal_grad.png}
\caption{\label{4.2.activity.kids.frontal} Combined frontal sensor activity from children in separate sensor groups. Top row: magnetometer activity; middle row: gradiometer activity; bottom two rows: equivalent results from the cluster analysis. Charts on the left side depict activity from the left hemisphere and vice versa.}
\end{center}
\end{figure}


\begin{figure}[!h]
\begin{center}
\includegraphics[width=0.49\textwidth]{pics/children_Left-temporal-magnetometer.png}
\includegraphics[width=0.49\textwidth]{pics/children_Right-temporal-magnetometer.png}
\includegraphics[width=0.49\textwidth]{pics/children_Left-temporal-gradiometer.png}
\includegraphics[width=0.49\textwidth]{pics/children_Right-temporal-gradiometer.png}
\includegraphics[width=0.49\textwidth]{pics/kids_Left-temporal_mag.png}
\includegraphics[width=0.49\textwidth]{pics/kids_Right-temporal_mag.png}
\includegraphics[width=0.49\textwidth]{pics/kids_Left-temporal_grad.png}
\includegraphics[width=0.49\textwidth]{pics/kids_Right-temporal_grad.png}
\caption{\label{4.2.activity.kids.temporal} Combined temporal sensor activity from children in separate sensor groups. Top row: magnetometer activity; middle row: gradiometer activity; bottom two rows: equivalent results from the cluster analysis. Charts on the left side depict activity from the left hemisphere and vice versa.}
\end{center}
\end{figure}


\begin{figure}[!h]
\begin{center}
\includegraphics[width=0.49\textwidth]{pics/children_Left-parietal-magnetometer.png}
\includegraphics[width=0.49\textwidth]{pics/children_Right-parietal-magnetometer.png}
\includegraphics[width=0.49\textwidth]{pics/children_Left-parietal-gradiometer.png}
\includegraphics[width=0.49\textwidth]{pics/children_Right-parietal-gradiometer.png}
\includegraphics[width=0.49\textwidth]{pics/kids_Left-parietal_mag.png}
\includegraphics[width=0.49\textwidth]{pics/kids_Right-parietal_mag.png}
\includegraphics[width=0.49\textwidth]{pics/kids_Left-parietal_grad.png}
\includegraphics[width=0.49\textwidth]{pics/kids_Right-parietal_grad.png}
\caption{\label{4.2.activity.kids.parietal} Combined parietal sensor activity from children in separate sensor groups. Top row: magnetometer activity; middle row: gradiometer activity; bottom two rows: equivalent results from the cluster analysis. Charts on the left side depict activity from the left hemisphere and vice versa.}
\end{center}
\end{figure}

\clearpage
For adults, clusters of significant differences were observed by left-temporal gradiometers and left-parietal magnetometers.
Left frontal gradiometers showed a weak positive effect between 131ms and 284ms ($p = 6.3\%$).
Left temporal gradiometers showed a positive effect between 257ms and 480ms ($p = 2.1\%$).
Left parietal magnetometers showed a positive effect between 618ms and 765ms ($p = 0.64\%$).
Left temporal magnetometers showed a weak negative effect between 1351 and 1491ms ($p = 8.4\%$).


\begin{figure}[!h]
\begin{center}
\includegraphics[width=0.49\textwidth]{pics/adults_Left-frontal-magnetometer.png}
\includegraphics[width=0.49\textwidth]{pics/adults_Right-frontal-magnetometer.png}
\includegraphics[width=0.49\textwidth]{pics/adults_Left-frontal-gradiometer.png}
\includegraphics[width=0.49\textwidth]{pics/adults_Right-frontal-gradiometer.png}
\includegraphics[width=0.49\textwidth]{pics/adults_Left-frontal_mag.png}
\includegraphics[width=0.49\textwidth]{pics/adults_Right-frontal_mag.png}
\includegraphics[width=0.49\textwidth]{pics/adults_Left-frontal_grad.png}
\includegraphics[width=0.49\textwidth]{pics/adults_Right-frontal_grad.png}
\caption{\label{4.2.activity.adults.frontal} Combined frontal sensor activity from adults in separate sensor groups. Top row: magnetometer activity; middle row: gradiometer activity; bottom two rows: equivalent results from the cluster analysis. Charts on the left side depict activity from the left hemisphere and vice versa.}
\end{center}
\end{figure}


\begin{figure}[!h]
\begin{center}
\includegraphics[width=0.49\textwidth]{pics/adults_Left-temporal-magnetometer.png}
\includegraphics[width=0.49\textwidth]{pics/adults_Right-temporal-magnetometer.png}
\includegraphics[width=0.49\textwidth]{pics/adults_Left-temporal-gradiometer.png}
\includegraphics[width=0.49\textwidth]{pics/adults_Right-temporal-gradiometer.png}
\includegraphics[width=0.49\textwidth]{pics/adults_Left-temporal_mag.png}
\includegraphics[width=0.49\textwidth]{pics/adults_Right-temporal_mag.png}
\includegraphics[width=0.49\textwidth]{pics/adults_Left-temporal_grad.png}
\includegraphics[width=0.49\textwidth]{pics/adults_Right-temporal_grad.png}
\caption{\label{4.2.activity.adults.temporal} Combined temporal sensor activity from adults in separate sensor groups. Top row: magnetometer activity; middle row: gradiometer activity; bottom two rows: equivalent results from the cluster analysis. Charts on the left side depict activity from the left hemisphere and vice versa.}
\end{center}
\end{figure}


\begin{figure}[!h]
\begin{center}
\includegraphics[width=0.49\textwidth]{pics/adults_Left-parietal-magnetometer.png}
\includegraphics[width=0.49\textwidth]{pics/adults_Right-parietal-magnetometer.png}
\includegraphics[width=0.49\textwidth]{pics/adults_Left-parietal-gradiometer.png}
\includegraphics[width=0.49\textwidth]{pics/adults_Right-parietal-gradiometer.png}
\includegraphics[width=0.49\textwidth]{pics/adults_Left-parietal_mag.png}
\includegraphics[width=0.49\textwidth]{pics/adults_Right-parietal_mag.png}
\includegraphics[width=0.49\textwidth]{pics/adults_Left-parietal_grad.png}
\includegraphics[width=0.49\textwidth]{pics/adults_Right-parietal_grad.png}
\caption{\label{4.2.activity.adults.parietal} Combined parietal sensor activity from adults in separate sensor groups. Top row: magnetometer activity; middle row: gradiometer activity; bottom two rows: equivalent results from the cluster analysis. Charts on the left side depict activity from the left hemisphere and vice versa.}
\end{center}
\end{figure}


The generally lower significance levels in adults imply an overall weaker impact of syntactic condition on sensor activity.
Syntactic effects in the left fronto-parietal region occurred earlier in adults (131-480ms) than in children (349-666ms).
Effects were much more lateralized in children, with a weak but distinct support from right parietal and temporal regions.
The strong effect in adults' left parietal regions between 618ms and 765ms is unparalleled in children.
These differences seem to promise a group effect, and will be resolved more accurately in the next section.

\clearpage