\section{Data acquisition}

\subsection {MEG}
MEG data were collected with an Elekta Neuromag VectorView\textsuperscript{\textregistered} MEG scanner in Bennewitz, at the development unit for Magnetoencephalography and Cortical Networks, Institute for Cognitive and Brain sciences, Leipzig, Germany.
The scanner comprised 306 MEG-channel sensors (102 magnetometers, 204 planar gradiometers).
It was passively shielded with two layers of \si{\micro}-metal and one layer of aluminium, dampening external signals between 40db (at 0.1Hz) and 80db (at 500Hz).
Sensors were tuned prior to each MEG recording session to limit noise levels to approximately $2.5 \frac{fT\sqrt{Hz}}{cm}$.
Sensors that became very noisy during a recording block would be individually re-tuned at the next inter-block break, either by using the fine-tuning options or the selective heating function.
Continuous MEG data were recorded at 1000 Hz sampling rate (330 Hz lowpass filter).

Prior to data acquisition, all metal and other potential sources of electromagnetic interference were removed from participants.
Quality of recording was confirmed by visual inspection of a live view of MEG recording before each session without the subject present.
Electro-oculogram (EOG) and electrocardiogram (ECG) time-series were recorded simultaneously with MEG to track potential noise sources and artifacts.
Five head position indicator (HPI) coils were attached to the participant's forehead and a Polhemus stylus and digitizer device were used to record the locations of fiducial points (right and left pre-auricular points (RPA, LPA) and nasion), the HPI coils, and between 150 and 200 extra digitizer points on the head surface.
Prior to the recording of each stimulus block, head location in the scanner was measured with an automatic process that detected the coils.
Continuous HPI recorded any head movements during data acquisition.

\subsection {MRI}
Anatomical magnetic resonance imaging (aMRI) data were collected with a 3.0 Tesla TIM Trio scanner, located at the Max-Planck-Institute for Cognitive and Brain sciences.
Two scans were acquired from each participant in one session: A T1-weighted scan and a T2-weighted scan.
The T1-weighted scan used the magnetization-prepared rapid gradient echo (MPRAGE, \cite{3.2.mprage}]) sequence (flip angle = $9\si{\degree}$, TR/TE/TI = $2300ms/2.96ms/900ms$).
This scan was oriented transverse (176 slices) with an isotropic resolution of 1mm.
The T2-weighted scan used the SPACE sequence by \cite{3.2.space} (flip angle = $120\si{\degree}$, TR/TE = $3200ms/402ms$).
This scan was oriented transverse (176 slices at 1mm) with an inplane resolution of 0.5mm x 0.5mm.
All scans used a 32-channel head coil for the acquisition.


