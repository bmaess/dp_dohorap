\section{Source-space activity}

\subsection{Comparison of means}
Localized activity from eight regions of interest (PAC, aSTS, aSTG, pSTS, pSTG, BA44, BA45 and BA6v) were examined for a syntactic effect.
Two different types of analysis were conducted: a cluster-based comparison and an interval-based comparison.

\paragraph{Cluster analysis}
The cluster comparison yielded only spurious effects in both groups ($p > 7\%$).

\paragraph{Interval analysis}
The interval comparison (0ms to 2200ms) yielded three distinct effect clusters in adults.

During the first time cluster, 0ms to 200ms, especially the right PAC showed a strong negative effect ($p = 0.5\%$, $t_{17} = -3.7$).
Other negative effects were visible in the left PAC ($p = 3.0\%$, $t_{17} = -2.9$) and the left BA45 ($p = 5.1\%$, $t_{17} = -2.6$).

The second time cluster, 400ms to 800ms, was signified by simultaneous effects in seven regions.
Both left and right PAC showed a positive effect ($p = 3.0\%$ and $p < 0.1\%$, $t_{17} = 2.9$ and $t_{17} = 4.8$).
The right pSTG showed negative effects, which grew more noticable during the progression of the time cluster.
In the first half of the cluster (400ms to 600ms), the condition effect started very weakly ($p = 7.9\%$, $t_{17} = -2.4$).
During the second half of the cluster (600ms to 800ms), the effect strengthened noticably ($p = 1.0\%$, $t_{17} = -3.9$).
The left aSTG showed a similar pattern: weak negative effect between 400ms and 600ms ($p = 2.1\%$, $t_{17} = -3.2$), which became noticably more distinct between 600ms and 800ms ($p = 0.4\%$, $t_{17} = -4.3$).
The left BA45 showed a constant and strong positive activation effect ($p = 0.6\%$, $t_{17} = 3.9$).
Otherwise, very weak positive effects ($p < 7\%$, $t_{17} = 3.0$) could be observed in the left BA6v (400ms to 600ms) and the left BA44 (600ms to 800ms).

The third time cluster, 1000ms to 1600ms, only showed effects in three regions.
The right PAC showed a very late, distinctly negative effect ($p = 0.9\%$, $t_{17} = -3.3$, between 1400ms and 1600ms).
Very weak positive effects ($p < 8\%$, $t_{17} = 2.4$) were observed in the right pSTG (1000ms to 1200ms) and the left aSTG (1400ms to 1600ms).
At the same time (1400ms to 1600ms), the left aSTS showed a very weak effect as well, but with opposite polarity ($p = 6.6\%$, $t_{17} = -3.0$).

\begin{figure}[h]
\begin{center}
\includegraphics[width=0.49\textwidth]{pics/signed-adults-PAC-lh.png}
\includegraphics[width=0.49\textwidth]{pics/signed-adults-PAC-rh.png}
\includegraphics[width=0.49\textwidth]{pics/signed-adults-aSTS-lh.png}
\includegraphics[width=0.49\textwidth]{pics/signed-adults-aSTS-rh.png}
\includegraphics[width=0.49\textwidth]{pics/signed-adults-aSTG-lh.png}
\includegraphics[width=0.49\textwidth]{pics/signed-adults-aSTG-rh.png}
\includegraphics[width=0.49\textwidth]{pics/signed-adults-BA45-lh.png}
\includegraphics[width=0.49\textwidth]{pics/signed-adults-BA45-rh.png}
\caption{\label{4.3.activity.adults.ventral} Combined activity from adults in separate cortical regions. Top row: PAC; second row: aSTS; third row: aSTG; bottom row: BA45. Charts on the left side depict activity from the left hemisphere and vice versa.}
\end{center}
\end{figure}


\begin{figure}[h]
\begin{center}
\includegraphics[width=0.49\textwidth]{pics/signed-adults-BA6v-lh.png}
\includegraphics[width=0.49\textwidth]{pics/signed-adults-BA6v-rh.png}
\includegraphics[width=0.49\textwidth]{pics/signed-adults-pSTS-lh.png}
\includegraphics[width=0.49\textwidth]{pics/signed-adults-pSTS-rh.png}
\includegraphics[width=0.49\textwidth]{pics/signed-adults-pSTG-lh.png}
\includegraphics[width=0.49\textwidth]{pics/signed-adults-pSTG-rh.png}
\includegraphics[width=0.49\textwidth]{pics/signed-adults-BA44-lh.png}
\includegraphics[width=0.49\textwidth]{pics/signed-adults-BA44-rh.png}
\caption{\label{4.3.activity.adults.ventral} Combined activity from adults in separate cortical regions. Top row: BA6v; second row: pSTS; third row: pSTG; bottom row: BA44. Charts on the left side depict activity from the left hemisphere and vice versa.}
\end{center}
\end{figure}


The interval analysis yielded no non-spurious effects ($p > 5\%$) from children.
This fact prompted a more extensive investigation.

\subsection{Post-hoc analysis}

Compared to the sensor-space analysis, source-space data yielded much more pronounced effects in adults.
Since this effect differentiation was the main goal of the localization process, these results were in line with my expectations.
However, localization of cortical activity in children failed to yield a similar improvement.

There were two major possibilities for these unexpected results.

First, the localization process could have been produced drastically worse inverse solutions for children than for adults.
Due to the hands-off approach of the generation of cortical surfaces, there are no parameters for the transfer of regional boundaries between the (adult) reference brain and each infant brain.
This automated process could have resulted in a systematically higher spatial error between my regional definitions and the actual functional regions in infants than in adults.
Less realistic regional definitions in children than in adults can result in unintended overlap between functional regions, and lead to a diminished experimental effect in each region.
It is also be possible that the automated segmentation process, that Freesurfer uses for extracting the cortex surface, is not optimized for infant brains.
An imprecise definition of the cortical surface would result in a skewed spatial location of the source dipoles, again causing regional overlapping and diminishing the experimental effect.

Second, 

