\section{Hypotheses}

\paragraph{Replicating earlier results}
Since similar paradigms have been used before, my intermediate results can be compared to previous findings.
Behaviorally, children are expected to perform worse than adults both in reaction time and accuracy.

I expect that syntax complexity has a positive impact on regional activity (i.e., higher activity with more complex syntax).

Based on previous fMRI studies, I expect a different locus of effect in both subject groups.
In adults, I expect the pSTG and the left BA44 to be affected most.
In children, the syntax effect will show mostly in the left pSTG and the left BA45.
Compared to adults, the left pSTG will be affected less strongly and less lateralized.

\paragraph{Explorative questions}
The condition effect has rarely been tested in isolation (i.e., without using semantic cues) in 10-year old children.
Based on the studies on slightly younger age groups, I speculate that neither response time nor accuracy should be affected by syntax complexity.

There is no existing study that resolves the temporal and spatial effect of complex syntax in different stages.
Based on existing theory, I expect an initial stage with bottom-up information flow, a second stage with consolidation between bottom-up and top-down processes, and a third stage with the response preparation.
These stages will show the syntax effect in the temporal cortex during the first stage, in both temporal and frontal cortex during the second stage, and show no effect during the third stage.

Based on the notion that adults rely more exclusively on the dorsal pathway 2 than children, I can form three hypotheses about the effect of complex syntax on interaction strength.
First, I expect a lower interaction strength for regions along the dorsal pathway 2, compared to regions along the ventral pathways. The opposite should be true for adults (pathway effect).
Second, I expect trials with complex syntax to cause a higher interaction strength in regions along the dorsal pathway 2, compared to trials with simple syntax (syntax x pathway effect).
Third, in a comparison of interaction strength within dorsal and ventral pathways, I expect complex syntax to cause a larger difference between pathways in adults than in children (syntax x pathway x group effect).

If the syntax effect turns out to be more lateralized in adults than in children, I expect a less lateralized information flow pathway in children as well.