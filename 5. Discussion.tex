\chapter{Discussion}

\paragraph{Research questions}

I expected, in comparison to the findings of \cite{2.1}, that my slightly altered paradigm would yield similar behavioral results.
These changes involved a syntax redesign and the omission of semantic cues.
My children performed with similar accuracy during subject-relative tasks (93\%, compared to 94\% in the reference study), but showed much lower accuracy during object-relative tasks (64\%, compared to 94\% in the reference study).
I considered these differences not to be in in conflict with the research paradigm, and speculated that they may have been the result of the omission of semantic cues.


I expected that the syntax condition had a positive impact on regional activity (i.e., elevated activity during tasks with object-relative tasks).

In children, there was a hint towards elevated activity in sensor space, indicated by a higher RMS in right temporal gradiometers (1384ms to 1663ms, $p = 3.8\%$).
This hint was not supported by the analysis of localized activity, which yielded only spurious effects.

In adults, two spatio-temporal regions hinted towards elevated activity, indicated by a higher RMS in left frontal (131ms to 284ms, $p = 6.3\%$) and left temporal (257ms to 480ms, $p = 2.1\%$) gradiometers. 
These hints were strenghtened by the analysis of localized activity.
In the left-frontal area, there was an early weak negative activation in the region BA45 (0 to 200ms, $p = 5.1\%$, $t_{17} = -2.6$).
In the medium-early left-temporal area, 

> Based on previous fMRI studies, I expected a different locus of effect in both subject groups.
> In adults, I expect the pSTG and the left BA44 to be affected most.
> In children, the syntax effect will show mostly in the left pSTG and the left BA45.
> Compared to adults, the left pSTG will be affected less strongly and less lateralized.


Interaction analysis:
- sensory feedforward and feedback model
- Trentool allows for the quantification of feedforward and feedback components
- epilepsy study shows bias towards short feedforward and long feedback propagation

\paragraph{What have similar studies found out?}

\paragraph{In which ways is my study limited, and what can it dare to say?}