\section{Source-space interactions}

Localized activity in eight cortical regions was analyzed for 28 possible signal interactions.
The following preliminary results were calculated with a reduced timing resolution of 2ms and a low upper time delay (20ms).
As such, these results may systematically underestimate both the effective transfer entropy and the delay of signal interactions.

First, transfer entropy was compared against surrogate data.
The surrogate test determined whether the transfered entropy (TE) was erroneously detected due to random coincidence.
This test revealed that all interactions were due to a real information transfer at a 5\% significance level (FDR-corrected with 28 comparisons).
At a 0.1\% level (Bonferroni-corrected with 28 comparisons), a significance pattern emerged.
Detailed significance results for all interactions at this threshold are visualized in figure \ref{4.4.networkgraph} in the form of arrow shading.
During the time window 0ms-300ms, 17 interactions were significant in the ORC condition and 15 in the SRC condition.
During the time window 400-800ms, 18 interactions were significant in the ORC condition and 17 in the SRC condition.
During the time window 1000ms-1600ms, 22 interactions were significant in the ORC condition and 21 in the SRC condition.

The effective TE was calculated from the difference between surrogate TE and raw TE.
Detailed TE values for all interactions are visualized in figure \ref{4.4.networkgraph} in the form of arrow thickness.
During the time window 0ms-300ms, the median TE was 7.6 in the ORC condition and 6.6 in the SRC condition.
During the time window 400-800ms, the median TE was 8.5 in the ORC condition and 8.3 in the SRC condition.
During the time window 1000ms-1600ms, the median TE was 8.7 in the ORC condition and 8.5 in the SRC condition.
The three biggest TE across all time windows were estimated for the interactions $PAC_{rh} \rightarrow PAC_{lh}$, $PAC_{rh} \rightarrow pSTG_{rh}$ and $PAC_{lh} \rightarrow pSTG_{rh}$.

The estimated delay between interactions was determined by selecting the signal shift that yielded the maximum TE.
Detailed delay values for all interactions are described in figure \ref{4.4.networkgraph} as number below each arrow.
The median delay was 11ms, with the 5\% smallest delays at 9ms and the 5\% largest delays at 14ms.
The longest delay was 15ms, from left BA6v to the right pSTG.
The shortest delay was 8ms, from left pSTG to the right pSTG.

\begin{figure}[h]
\begin{center}
\vspace{7mm}
\includegraphics[width=\textwidth]{pics/TE_results.png}
\caption{\label{4.4.networkgraph} Graph of the estimated transfer entropy and signal delay between selected cortical regions. Arrow labels refer to the estimated signal delay and the effective transfer entropy. Arrow thickness is depicted relative to the effective transfer entropy. Values are averages over all time windows and conditions.}
\end{center}
\end{figure}

Second, the impact of the syntax condition on the transfer entropy was determined with a group-level cluster comparison.
During the time window 0ms-300ms, a weak effect ($p = 5.8\%$, $\DELTA TE = 4.0$) appeared on the interaction $BA45_{lh} \rightarrow BA44_{lh}$.
During the time window 400-800ms, a weak effect ($p = 8.0\%$, $\DELTA TE = 3.0$) appeared on the interaction $pSTG_{lh} \rightarrow pSTG_{rh}$.
During the time window 1000ms-1600ms, no effects appeared on any interaction.
