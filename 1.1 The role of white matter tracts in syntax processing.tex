\chapter{Introduction}

\section{The role of white matter tracts in syntax processing}

\subsection{Exploration of white matter tracts}
The human brain can be regarded as an information-processing system, processing and storing environmental stimuli in order to create meaningful actions. 
In a very simplified information-processing perspective, it can be reduced to two functional components: computational units in the grey matter and signal relays in the white matter.
Contrary to other common information-theoretical systems, there is no fundamental distinction between hardware and software in the brain.
Both structural elements (e.g., dendritic spines) and computational units (e.g., astroglial cells) are part of an inseparable process.
The fact that anatomical features can't be clearly mapped to a functional role violates the naïve reductionist approach, and puts cognitive science more in line with other fields that explore dynamical systems, such as physics, theoretical computer science or meteorology.
Although the exploration of anatomic features (e.g., the classification of cortical regions based on myeloarchitecture ) led to several ground-breaking discoveries in psychology, these findings tend to serve as a base for more specialized studies, rather than creating a framework for understanding cognitive processes.

In recent years, another antomic feature became accessible for exploration due to advances in neuroimaging.
The cerebral white matter, containing mainly myelinated axons and glial cells, was suspected to play a role in signal transmission as early as 1710 [1.1.history].
In the writings of Pourfoir de Petit (1664-1741), nerve fiber bundles were theorized to transmit animal spirits from the cortex to the medulla oblongata.
The complete signal transmission chain, continuing the activation from the medulla to the extremities via the spinal cord was made by Emanuel Swedenborg between 1738 and 1744 [1.1.history].
However, during the next 250 years, the exploration of the role of cerebral fiber tracts was stifled by the lack of a non-invasive, large-scale measurement method.
These circumstances apruptly changed in 1972 [1.1.history], when the discovery of the autoradiographic method allowed tracing the precise interneural connections with radioactive markers.
This method was very widely used to fill existing knowledge gaps in cognitive science, but one fundamental flaw has prevented its application in humans: not only is the tracer very invasive, but the subject needs to be killed and dissected in the course of the experiment.
This drawback has prevented the study of the role of human fiber bundles, and not all knowledge gaps could be filled with equivalent results from lab animals.
One of the most important set of fiber bundles that eluded autoradiographic study was the human language network.
Only with the advent of diffusion imaging


There are a few biophysical properties that pose unique constraints to this perspective.
Connections between computational units (neuronal clusters) are commonly realized as nerve fibers, usually in the form of singular, branchless axons.
The diameter of each axon directly influences its transmission speed.
However, due to the limited amount of space in the skull cavity, the brain is fundamentally volume-constrained.
New connections can therefore only be created in two ways: either another connection is reduced in volume, allowing the 

The most important task of the brain, learning, requires the creation of new connections between existing neuronal clusters[1.1.whitematter].

- fibers are expensive, brain tries to save space and energy
- most connections are local, "small world"
- local connections mean that neighboring neuronal clusters have similar tasks
- local connections cause gyration
- also lead to automatic grouping of task-related neuronal clusters into task-specialized cortical regions

- biophysical connection between bandwidth and volume
- learning requires shrinking old fibers and increased bandwidth compression (for both old and new fibers)
- volume-constrained optimization process: pressure on long-distance

- existing long-distance fiber bundles, although requiring a large volume, have survived this optimization
- therefore, bandwidth compression and signal complexity must be very high on these fiber bundles
- end points of bundles play role of information integration and (de-)compression
- fiber bundles, in the framework of the connectome, have only recently been explored


\subsection{Language network}
- important fiber tracts for language: ECFS and UF
- two information processing streams along two fiber tracts
- cortial regions at the end points of the tracts
- role of each cortical region
- pSTG: role not yet known
- pSTG is known to be used in SRC/ORC grammar tasks (from fMRI)
- information streams from and to pSTG -> indication of its role

Dorsal \& Ventral streams
The ventral stream uses two major long-distance fiber tracts: ECFS and UF
Ventral pathway I (ECFS): STG to BA45
Ventral pathway II (UF): aSTG to FOP
The dorsal stream uses two major long-distance fiber tracts: The arcuate fascicle and the superior longitudinal fascicle.
Dorsal pathway I (SLF): pSTG to premotor cortex in BA6
Dorsal pathway II (AF): pSTG to BA44

During sentence processing, the IFG and posterior superior temporal regions have been proposed to exchange information via
 the dorsally located Arcuate Fasciculus/Superior Longitudinal Fasciculus (AF/SLF; Friederici, 2011; Meyer, Obleser, Anwander, \& Friederici, 2012; Wilson et al., 2011). This proposal still awaits direct causal support, since it is based either on tractography in healthy participants (Catani, Jones, \& ffytche, 2005; Friederici, Bahlmann, Heim, Schubotz, \& Anwander, 2006; Glasser \& Rilling, 2008; Meyer et al., 2012; Parker et al., 2005; Saur et al., 2010; Weiller, Musso, Rijntjes, \& Saur, 2009) or data from sentence-processing-impaired conduction aphasics whose lesions mostly involve both gray- and white-matter damage (Baldo, Klostermann, \& Dronkers, 2008; Buchsbaum et al., 2011; Friedmann \& Gvion, 2003) and primary progressive aphasics (Galantucci et al., 2011; Wilson et al., 2010, 2011). The AF/SLF has been taken to be particularly relevant for the
 processing of sentences with complex word orders (Friederici,
 2011; Wilson et al., 2011).

