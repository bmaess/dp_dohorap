\chapter{Introduction}

\section{The role of white matter tracts in syntax processing}

\subsection{White fiber tracts}
Brain tries to save space and energy
Connections cost both space and energy, so brain is largely locally organized
Long-distance connections cost much more space and energy than local connections, while not transferring more information
The fact that long-distance connections exist means the transferred information is especially important and/or highly compressed

\subsection{Language network}
Dorsal \& Ventral streams
The ventral stream uses two major long-distance fiber tracts: ECFS and UF
Ventral pathway I (ECFS): STG to BA45
Ventral pathway II (UF): aSTG to FOP
The dorsal stream uses two major long-distance fiber tracts: The arcuate fascicle and the superior longitudinal fascicle.
Dorsal pathway I (SLF): pSTG to premotor cortex in BA6
Dorsal pathway II (AF): pSTG to BA44

During sentence processing, the IFG and posterior superior
 temporal regions have been proposed to exchange information via
 the dorsally located Arcuate Fasciculus/Superior Longitudinal Fasci-
 culus (AF/SLF; Friederici, 2011; Meyer, Obleser, Anwander, \&
 Friederici, 2012; Wilson et al., 2011). This proposal still awaits direct
 causal support, since it is based either on tractography in healthy
 participants (Catani, Jones, \& ffytche, 2005; Friederici, Bahlmann,
 Heim, Schubotz, \& Anwander, 2006; Glasser \& Rilling, 2008; Meyer
 et al., 2012; Parker et al., 2005; Saur et al., 2010; Weiller, Musso,
 Rijntjes, \& Saur, 2009) or data from sentence-processing-impaired
 conduction aphasics whose lesions mostly involve both gray- and
 white-matter damage (Baldo, Klostermann, \& Dronkers, 2008; Buchsbaum et al., 2011; Friedmann \& Gvion, 2003) and primary
 progressive aphasics (Galantucci et al., 2011; Wilson et al., 2010,
 2011). The AF/SLF has been taken to be particularly relevant for the
 processing of sentences with complex word orders (Friederici,
 2011; Wilson et al., 2011).

