\section{Experimental paradigm}

My goal was to investigate the question if dorsal pathway 2 is carrying top-down or bottom-up signals while parsing complex syntax.
To reach this goal, there were four questions that needed to be answered.


The first question was which subjects to use.

Existing studies have been predominantly performed on two groups of subjects: adults and infants.

For adults, current evidence indicates that complex syntactic processing involves a combination of the left pSTG/STS and the pars opercularis of the left inferior frontal gyrus in Broca's area.

In infants, complex syntax processing seems to rely more on the ventral pathway.
This different processing approach explains why children are prone to be biased by semantic cues and why their BA45 shows more involvement while parsing complex syntax.
Diffusion imaging studies have explained this systematic bias by showing that the arcuate fascicle, which is used by dorsal pathway 2, isn't fully developed until early childhood.

In order to expand the existing knowledge about functional development, I required an age group that was already able to solve complex grammar, but whose dorsal pathway 2 was not entirely developed yet.
10-year old children are the oldest group which still performs worse than adults.
I selected subjects who were as mature as possible in order to maximize their patience (useful for motion-free anatomical MRI scans) and their endurance for long behavioral sessions (useful for maximizing the amount of trials for my experiment).

Additionally, I selected a similarly sized adult group for analyzing signal interactions in the adult brain.
Ideally, the differences in processing between adults and children can provide new insights into the developmental process of the dorsal pathway 2.


The second question was which task these subjects should be solving.

The literature provides an established and robust way to isolate the dorsal pathway 2 in psychometric experiments.
Object-relative clauses are known to affect regions connected to this pathway more then subject-relative clauses.
Evidence for the locations of affected regions comes predominantly from fMRI studies.
One of those studies \cite{1.3.Constable} used an actor-undergoer paradigm with sentences that differed only in their syntax order:


\begin{center}
\emph{"'The biologist - who showed the video - studied the snake."'}
\emph{"'The biologist - who the video showed - studied the snake."'}
\end{center}


This paradigm produced a syntax effect in the left BA40 and BA49, as well as in both BA44/BA45 and ventral premotor cortices.

This process was later refined \cite{1.3.Bornkessel} by using a grammatical peculiarity in the German language: changing between subject- and object-relative clauses is possible even with a fixed word order, by changing only a few suffices and pronouns.
In this study, cases E and F contained unambiguous subject- and object-relative clauses with an active verb.


\begin{center}
\emph{"'Gestern wurde erzählt, dass der Junge den Lehrern hilft."'}
\emph{"'Gestern wurde erzählt, dass dem Junge die Lehrer helfen."'}
\end{center}


The fMRI-based comparison between these cases yielded the following effect locations:
Left hemisphere: IFG (more specifically, the pars opercularis), pSTG, pSTS, inferior frontal junction, ventral premotor cortex.
Both hemispheres: interparietal sulcus.

I adapted this strategy to create German sentences with a very similar structure.
This process is described in detail in chapter \ref{3.1.stimuli.auditory}.


The third question was how to measure and interpret the evoked cerebral activity.
This topic will be covered in detail in section \ref{1.5}.


The fourth and final question was which results to expect from this study.
