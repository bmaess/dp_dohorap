1.3 Experimental paradigm

Task: Exploring role and timing of pSTG/pSTS activity in complex syntax processing
Actor/Undergoer setup
Object-/Subject-relative clauses

Sentences are comparable to Constable, 2004:
Subject/object-relative clauses
"The biologist — who showed the video — studied the snake."
"The biologist — who the video showed — studied the snake."
fMRI effect in:
Left: BA40, 44/45, 39, premotor cortex
Right: BA44/45, premotor cortex

Also comparable to Bornkessel, 2006:
specifically, cases E and F (unambiguous subject- and object-relative clauses with active verb)
significant fMRI effect in:
left: IFG (pars opercularis), pSTG, pSTS, inferior frontal junction, ventral premotor cortex, interparietal sulcus
right: interparietal sulcus

Repeated-measures design
Stationarity assumption

[]
" information processing can be broken down into the three components of
 information storage, information transfer, and information modification [1–4]. These components can be
 easily identified in theoretical, or technical, information processing systems, such as ordinary comput-
 ers, based on the specialized machinery for and the spatial separation of these component functions. In
 these examples, a separation of the components of information processing via a specialized mathematical
 formalism seems almost superfluous. However, in biological systems in general, and in the brain in par-
 ticular, we deal with a form of distributed information processing based on a large number of interacting
 agents (neurons), and each agent at each moment in time subserves any of the three component functions
 to a varying degree. In neural systems it is indeed crucial to understand where and when information
 storage, transfer and modification take place, to constrain possible algorithms run by the system. While
 there is still a struggle to properly define a measure for information modification [5,6] and its proper mea-
 sure [7–10], well established measures for (local active) information storage [11], information transfer [12],
 and its localization in time [13] exist.
 Especially the measure for information transfer, transfer entropy (TE), has seen a dramatic surge
 of interest in neuroscience [14–31], physiology [32–34], and other fields [5, 13, 26, 35, 36]. Nevertheless,
 conceptual and practical problems still exist. On the conceptual side, information transfer has been for a
 while confused with causal interactions, and only some recent studies [37–39] made clear that there can be
 no one-to-one mapping between causal interactions and information transfer, because causal interactions
 will subserve all three components of information processing (transfer, storage, modification). However,
 it is information transfer, rather than causal interactions, we might be interested in when trying to
 understand a computational process in the brain [38]."

The 