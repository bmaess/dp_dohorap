
\chapter{Results}\label{results}

\section{Behavioral results}

If not noted otherwise, two comma-separated values in brackets describe the upper and lower values of a 95\% confidence interval.

\subsection{Response times}

A Shapiro-Wilk test was conducted to determine if individual response times were normal distributed.
All subjects failed this test (p < 0.001), indicating a strong deviation from normality.
Therefore, I represented individual response times by their median.

Children needed a median time of 1.91s (1.64s, 2.19s) to respond to object-relative clauses.
For subject-relative clauses, they needed 1.97s (1.69s, 2.25s).
Adults needed a median time of 1.51s (1.28s, 1.73s) to respond to object-relative clauses.
For subject-relative clauses, they needed 1.60s (1.37s, 1.82s).

A Shapiro-Wilk test determined that response time data was normal distributed with a probability between 1.5\% and 27\%.
A Levene's test determined that the probability of median response times being normal distributed was ?\%.
Supported by these findings, the response times were included into the ANOVA.

\subsection{Response accuracy}

For the analysis of variance (ANOVA), all data must be normal distributed with equal variance.
A Shapiro-Wilk test determined that the probability of accuracy data being normal distributed was between 2.0 and 20.3\%.
A Levene's test yielded that the probability that accuracy data were distributed with equal variance was less than p = 0.1\%.

The ANOVA is known to be robust for considerable deviations from the normal distribution.
However, it is highly vulnerable to violation the assumption of equal variances.
To meet this requirement, I transformed the accuracy data with the inverse sigmoid function.
This procedure, however, created singularities in some extreme cases, i.e., when a subject performed with a 100\% accuracy rate.
To prevent this issue, I added a single incorrect trial to every subject's performance for the following analysis.

After the transformation, the same tests as before were conducted.
The probability for transformed accuracy data being normal distributed was between 0.5\% and 27\%.
The probability for transformed accuracy data being distributed with equal variance was p = 72.2\%.
Supported by these findings, the transformed accuracy data was included in the ANOVA.

\subsection{Analysis of combined performance data}

Two ANOVA were conducted with the transformed accuracy data and the median response times.
Each subject provided one data point for each metric.
Data were analyzed with a group x condition design.
Accuracy estimates were transformed back with the sigmoid function $r = \frac{1}{1+e^{\hat{-r}}}$.

Children responded 0.39s slower than adults (1.94s vs. 1.55s).
This difference was significant ($F_{56} = 9.4$, $p = 0.3\%$).

Children responded with an average accuracy of 93.8\% (92.2\%, 95.0\%).
Adults performed much better, with an average accuracy of 97.9\% (97.5\%, 98.3\%).
This difference was highly significant ($F_{56} = 52$, $p = 1.6*10^{-9}$).

Sentence condition had no impact on median response times ($F = 0.33$, $p = 57\%$) or on response accuracy ($F = 1.3$, $p = 26\%$).
There was no interaction effect between group and sentence condition ($F < 0.1$, $p > 80\%$).

\section{Sensor-space activity}

We computed average event-related fields (ERF) for each subject and sensor region.
Activity from these ERF was selected with two different types of time windows.

A positive effect indicates that activity evoked by object-relative clauses was more positive than activity evoked by subject-relative clauses. In the case of gradiometers, "'more positive"' means a higher regional RMS. With localized data, "'more positive"' means a higher z-score from the sLORETA source reconstruction.


\subsection{Interval analysis}
For this analysis, sensor activity from separate regions and hemispheres was compared blindly between 0 and 2200ms after onset in 200ms intervals.

After FDR-correction for 10 comparisons, no sensor region showed any non-spurious effect in either group ($p > 3\%$).

\subsection{Cluster analysis}
For this analysis, activity was compared between conditions using a temporal cluster analysis.

For children, significant differences were observed in the following sensor groups:
Right parietal magnetometers showed a negative effect between 159ms and 374ms ($p = 1.4\%$).
Left frontal magnetometers showed a negative effect between 375ms and 666ms ($p = 1.8\%$).
Left temporal magnetometers showed a negative effect between 349ms and 627ms ($p = 0.8\%$).
Right temporal gradiometers showed a positive effect between 1384ms and 1663ms ($p = 3.8\%$).

\begin{figure}[!h]
\begin{center}
\includegraphics[width=0.49\textwidth]{pics/children_Left-frontal-magnetometer.png}
\includegraphics[width=0.49\textwidth]{pics/children_Right-frontal-magnetometer.png}
\includegraphics[width=0.49\textwidth]{pics/children_Left-frontal-gradiometer.png}
\includegraphics[width=0.49\textwidth]{pics/children_Right-frontal-gradiometer.png}
\includegraphics[width=0.49\textwidth]{pics/kids_Left-frontal_mag.png}
\includegraphics[width=0.49\textwidth]{pics/kids_Right-frontal_mag.png}
\includegraphics[width=0.49\textwidth]{pics/kids_Left-frontal_grad.png}
\includegraphics[width=0.49\textwidth]{pics/kids_Right-frontal_grad.png}
\caption{\label{4.2.activity.kids.frontal} Combined frontal sensor activity from children in separate sensor groups. Top row: magnetometer activity; middle row: gradiometer activity; bottom two rows: equivalent results from the cluster analysis. Charts on the left side depict activity from the left hemisphere and vice versa.}
\end{center}
\end{figure}


\begin{figure}[!h]
\begin{center}
\includegraphics[width=0.49\textwidth]{pics/children_Left-temporal-magnetometer.png}
\includegraphics[width=0.49\textwidth]{pics/children_Right-temporal-magnetometer.png}
\includegraphics[width=0.49\textwidth]{pics/children_Left-temporal-gradiometer.png}
\includegraphics[width=0.49\textwidth]{pics/children_Right-temporal-gradiometer.png}
\includegraphics[width=0.49\textwidth]{pics/kids_Left-temporal_mag.png}
\includegraphics[width=0.49\textwidth]{pics/kids_Right-temporal_mag.png}
\includegraphics[width=0.49\textwidth]{pics/kids_Left-temporal_grad.png}
\includegraphics[width=0.49\textwidth]{pics/kids_Right-temporal_grad.png}
\caption{\label{4.2.activity.kids.temporal} Combined temporal sensor activity from children in separate sensor groups. Top row: magnetometer activity; middle row: gradiometer activity; bottom two rows: equivalent results from the cluster analysis. Charts on the left side depict activity from the left hemisphere and vice versa.}
\end{center}
\end{figure}


\begin{figure}[!h]
\begin{center}
\includegraphics[width=0.49\textwidth]{pics/children_Left-parietal-magnetometer.png}
\includegraphics[width=0.49\textwidth]{pics/children_Right-parietal-magnetometer.png}
\includegraphics[width=0.49\textwidth]{pics/children_Left-parietal-gradiometer.png}
\includegraphics[width=0.49\textwidth]{pics/children_Right-parietal-gradiometer.png}
\includegraphics[width=0.49\textwidth]{pics/kids_Left-parietal_mag.png}
\includegraphics[width=0.49\textwidth]{pics/kids_Right-parietal_mag.png}
\includegraphics[width=0.49\textwidth]{pics/kids_Left-parietal_grad.png}
\includegraphics[width=0.49\textwidth]{pics/kids_Right-parietal_grad.png}
\caption{\label{4.2.activity.kids.parietal} Combined parietal sensor activity from children in separate sensor groups. Top row: magnetometer activity; middle row: gradiometer activity; bottom two rows: equivalent results from the cluster analysis. Charts on the left side depict activity from the left hemisphere and vice versa.}
\end{center}
\end{figure}

\clearpage
For adults, clusters of significant differences were observed by left-temporal gradiometers and left-parietal magnetometers.
Left frontal gradiometers showed a weak positive effect between 131ms and 284ms ($p = 6.3\%$).
Left temporal gradiometers showed a positive effect between 257ms and 480ms ($p = 2.1\%$).
Left parietal magnetometers showed a positive effect between 618ms and 765ms ($p = 0.64\%$).
Left temporal magnetometers showed a weak negative effect between 1351 and 1491ms ($p = 8.4\%$).


\begin{figure}[!h]
\begin{center}
\includegraphics[width=0.49\textwidth]{pics/adults_Left-frontal-magnetometer.png}
\includegraphics[width=0.49\textwidth]{pics/adults_Right-frontal-magnetometer.png}
\includegraphics[width=0.49\textwidth]{pics/adults_Left-frontal-gradiometer.png}
\includegraphics[width=0.49\textwidth]{pics/adults_Right-frontal-gradiometer.png}
\includegraphics[width=0.49\textwidth]{pics/adults_Left-frontal_mag.png}
\includegraphics[width=0.49\textwidth]{pics/adults_Right-frontal_mag.png}
\includegraphics[width=0.49\textwidth]{pics/adults_Left-frontal_grad.png}
\includegraphics[width=0.49\textwidth]{pics/adults_Right-frontal_grad.png}
\caption{\label{4.2.activity.adults.frontal} Combined frontal sensor activity from adults in separate sensor groups. Top row: magnetometer activity; middle row: gradiometer activity; bottom two rows: equivalent results from the cluster analysis. Charts on the left side depict activity from the left hemisphere and vice versa.}
\end{center}
\end{figure}


\begin{figure}[!h]
\begin{center}
\includegraphics[width=0.49\textwidth]{pics/adults_Left-temporal-magnetometer.png}
\includegraphics[width=0.49\textwidth]{pics/adults_Right-temporal-magnetometer.png}
\includegraphics[width=0.49\textwidth]{pics/adults_Left-temporal-gradiometer.png}
\includegraphics[width=0.49\textwidth]{pics/adults_Right-temporal-gradiometer.png}
\includegraphics[width=0.49\textwidth]{pics/adults_Left-temporal_mag.png}
\includegraphics[width=0.49\textwidth]{pics/adults_Right-temporal_mag.png}
\includegraphics[width=0.49\textwidth]{pics/adults_Left-temporal_grad.png}
\includegraphics[width=0.49\textwidth]{pics/adults_Right-temporal_grad.png}
\caption{\label{4.2.activity.adults.temporal} Combined temporal sensor activity from adults in separate sensor groups. Top row: magnetometer activity; middle row: gradiometer activity; bottom two rows: equivalent results from the cluster analysis. Charts on the left side depict activity from the left hemisphere and vice versa.}
\end{center}
\end{figure}


\begin{figure}[!h]
\begin{center}
\includegraphics[width=0.49\textwidth]{pics/adults_Left-parietal-magnetometer.png}
\includegraphics[width=0.49\textwidth]{pics/adults_Right-parietal-magnetometer.png}
\includegraphics[width=0.49\textwidth]{pics/adults_Left-parietal-gradiometer.png}
\includegraphics[width=0.49\textwidth]{pics/adults_Right-parietal-gradiometer.png}
\includegraphics[width=0.49\textwidth]{pics/adults_Left-parietal_mag.png}
\includegraphics[width=0.49\textwidth]{pics/adults_Right-parietal_mag.png}
\includegraphics[width=0.49\textwidth]{pics/adults_Left-parietal_grad.png}
\includegraphics[width=0.49\textwidth]{pics/adults_Right-parietal_grad.png}
\caption{\label{4.2.activity.adults.parietal} Combined parietal sensor activity from adults in separate sensor groups. Top row: magnetometer activity; middle row: gradiometer activity; bottom two rows: equivalent results from the cluster analysis. Charts on the left side depict activity from the left hemisphere and vice versa.}
\end{center}
\end{figure}


The generally lower significance levels in adults imply an overall weaker impact of syntactic condition on sensor activity.
Syntactic effects in the left fronto-parietal region occurred earlier in adults (131-480ms) than in children (349-666ms).
Effects were much more lateralized in children, with a weak but distinct support from right parietal and temporal regions.
The strong effect in adults' left parietal regions between 618ms and 765ms is unparalleled in children.
These differences seem to promise a group effect, and will be resolved more accurately in the next section.

\clearpage\section{Source-space activity}\label{4.3}

\subsection{Comparison of means}
Localized activity from eight regions of interest (PAC, aSTS, aSTG, pSTS, pSTG, BA44, BA45 and BA6v) were examined for a syntactic effect.
Two different types of analysis were conducted: a cluster-based comparison and an interval-based comparison.

\paragraph{Cluster analysis}
The cluster comparison yielded only spurious effects in both groups ($p > 7\%$).

\paragraph{Interval analysis}
The interval comparison (0ms to 2200ms) yielded three distinct effect clusters in adults.

During the first time cluster, 0ms to 200ms, especially the right PAC showed a strong negative effect ($p = 0.5\%$, $t_{17} = -3.7$).
Other negative effects were visible in the left PAC ($p = 3.0\%$, $t_{17} = -2.9$) and the left BA45 ($p = 5.1\%$, $t_{17} = -2.6$).

The second time cluster, 400ms to 800ms, was signified by simultaneous effects in seven regions.
Both left and right PAC showed a positive effect ($p = 3.0\%$ and $p < 0.1\%$, $t_{17} = 2.9$ and $t_{17} = 4.8$).
The right pSTG showed negative effects, which grew more noticable during the progression of the time cluster.
In the first half of the cluster (400ms to 600ms), the condition effect started very weakly ($p = 7.9\%$, $t_{17} = -2.4$).
During the second half of the cluster (600ms to 800ms), the effect strengthened noticably ($p = 1.0\%$, $t_{17} = -3.9$).
The left aSTG showed a similar pattern: weak negative effect between 400ms and 600ms ($p = 2.1\%$, $t_{17} = -3.2$), which became noticably more distinct between 600ms and 800ms ($p = 0.4\%$, $t_{17} = -4.3$).
The left BA45 showed a constant and strong positive activation effect ($p = 0.6\%$, $t_{17} = 3.9$).
Otherwise, very weak positive effects ($p < 7\%$, $t_{17} = 3.0$) could be observed in the left BA6v (400ms to 600ms) and the left BA44 (600ms to 800ms).

The third time cluster, 1000ms to 1600ms, only showed effects in three regions.
The right PAC showed a very late, distinctly negative effect ($p = 0.9\%$, $t_{17} = -3.3$, between 1400ms and 1600ms).
Very weak positive effects ($p < 8\%$, $t_{17} = 2.4$) were observed in the right pSTG (1000ms to 1200ms) and the left aSTG (1400ms to 1600ms).
At the same time (1400ms to 1600ms), the left aSTS showed a very weak effect as well, but with opposite polarity ($p = 6.6\%$, $t_{17} = -3.0$).

\begin{figure}[H]
\begin{center}
\includegraphics[width=0.49\textwidth]{pics/signed-adults-PAC-lh.png}
\includegraphics[width=0.49\textwidth]{pics/signed-adults-PAC-rh.png}
\includegraphics[width=0.49\textwidth]{pics/signed-adults-aSTS-lh.png}
\includegraphics[width=0.49\textwidth]{pics/signed-adults-aSTS-rh.png}
\includegraphics[width=0.49\textwidth]{pics/signed-adults-aSTG-lh.png}
\includegraphics[width=0.49\textwidth]{pics/signed-adults-aSTG-rh.png}
\includegraphics[width=0.49\textwidth]{pics/signed-adults-BA45-lh.png}
\includegraphics[width=0.49\textwidth]{pics/signed-adults-BA45-rh.png}
\caption{\label{4.3.activity.adults.ventral} Combined activity from adults in separate cortical regions. Top row: PAC; second row: aSTS; third row: aSTG; bottom row: BA45. Charts on the left side depict activity from the left hemisphere and vice versa.}
\end{center}
\end{figure}


\begin{figure}[H]
\begin{center}
\includegraphics[width=0.49\textwidth]{pics/signed-adults-BA6v-lh.png}
\includegraphics[width=0.49\textwidth]{pics/signed-adults-BA6v-rh.png}
\includegraphics[width=0.49\textwidth]{pics/signed-adults-pSTS-lh.png}
\includegraphics[width=0.49\textwidth]{pics/signed-adults-pSTS-rh.png}
\includegraphics[width=0.49\textwidth]{pics/signed-adults-pSTG-lh.png}
\includegraphics[width=0.49\textwidth]{pics/signed-adults-pSTG-rh.png}
\includegraphics[width=0.49\textwidth]{pics/signed-adults-BA44-lh.png}
\includegraphics[width=0.49\textwidth]{pics/signed-adults-BA44-rh.png}
\caption{\label{4.3.activity.adults.ventral} Combined activity from adults in separate cortical regions. Top row: BA6v; second row: pSTS; third row: pSTG; bottom row: BA44. Charts on the left side depict activity from the left hemisphere and vice versa.}
\end{center}
\end{figure}

\clearpage

The interval analysis yielded no non-spurious effects ($p > 5\%$) from children.
This fact prompted a more extensive investigation.

\subsection{Post-hoc analysis}

Compared to the sensor-space analysis, source-space data yielded much more pronounced effects in adults.
Since this effect differentiation was the main goal of the localization process, these results were in line with my expectations.
However, localization of cortical activity in children failed to yield a similar improvement.
The localization procedure was identical for both groups, so no procedural difference could have affected the results.

That left two major possibilities for these unexpected results.

\paragraph{Possible explanations}
First, the localization process may have produced drastically worse inverse solutions for children than for adults.
Due to the hands-off approach of the generation of cortical surfaces, there are no parameters for the transfer of regional boundaries between the (adult) reference brain and each infant brain.
This automated process could have resulted in a systematically higher spatial error between my regional definitions and the actual functional regions in infants than in adults.
Less realistic regional definitions in children than in adults can result in unintended overlap between functional regions, and lead to a diminished experimental effect in each region.
It is also be possible that the automated segmentation process, that Freesurfer uses for extracting the cortex surface, is not optimized for infant brains.
An imprecise definition of the cortical surface would result in a skewed spatial location of the source dipoles, again causing regional overlapping and diminishing the experimental effect.
If the localization performed drastically worse for children, the extraction of relevant time intervals and regional connections from the conditional effect would be impossible.

Second, it is possible that the cognitive strategies were stable within each subject, but those of my young subjects were  more diverse than those of my adults.
Both cluster analysis and interval analysis are based on the assumption that all subjects in one group use the same cognitive strategy.
If this assumption is true for adults, but not for children, different processing strategies could cancel each other out and produce no effect.
Both the average localized activity and pooled single trials would be vulnerable to this violation.

Third, it is possible that cognitive strategies varied within subjects for children, but not for adults.
This circumstance would lead the conditional effect to appear in different regions over the course of the task.
Mean regional activity would be blurred because the stationarity assumption would be violated, and the overall effect would be diminshed.

\paragraph{Exploring the explanations}
Unfortunately, in order to compare the first possibility, a ground truth for the dipole locations and activations would be necessary.
This type of reference data is only available in phantom models, therefore the first possibility remains elusive for direct statistical tests.
However, since neither set of localization software was designed for processing infant brains, this possibility remains the  default explanation.

To decide whether the second possibility was responsible for the skewed results, I conducted two tests:
First, if the cognitive processes were much more varied in children than in adults, this circumstance should be reflected in a more homogenuous evoked activity within the adults.
To test this hypothesis, I calculated the individual deviation from the group-average activity and compared deviations between both groups.

Second, if the cognitive strategy was less stable in children than in adults, localized activity in children should show more intra-subject variation.
To test this hypothesis, I calculated the individual deviation from the individual-average activity and compared deviations between both groups.

\paragraph{Test results}
For the first test, localized pooled activity (evoked by object-relative clauses) was selected from the time window 0ms to 2200ms.
The mean activity was computed from this selected activity from all subjects within each group.
The difference between individual and mean activity was then computed for each cortical region, yielding a time series.
Each time series was reduced to a single value by computing the variance over time.
Variances from both groups were tested for systematic differences with Welch's test.
The Welch test doesn't assume equal variances and allows for unequal sample sizes.
It was implemented in Python with the function \emph{scipy.stats.ttest\_ind()}.
Variances from both groups differed only weakly ($p \approx 5\%$, $t \approx -2.8$) after FDR correction with 8 comparisons.
There was a considerable bias in six regions (all expect for pSTS and pSTG) towards higher group-level variances in adults.
This finding weakened the second possibility (stronger inter-subject variation in children), and indirectly supported the first possibility (better localization in adults).

The second test was based on the same data set as the first test.
Reference activity was calculated by computing the mean of all trials from one subject.
Then, the difference between single trials and the reference trial were computed and reduced by calculating the variance over time.
A single value for each subject was determined by computing the mean of all subject-specific variances.
Again, Welch's test compared individual variance scores between both groups.
The FDR-corrected results (8 comparisons) yielded no significant ($p > 50\%$, $t \approx -1.0$) differences.
This finding weakened the third possibility (stronger intra-subject variation in children), and indirectly supported the first possibility (better localization in adults).

\paragraph{Conclusion}
The deterioration of the syntactical effect in children before and after the source localization can not be explained by a wider spectrum of cognitive strategies.
If anything, adults showed a wider, not a smaller variety of localized activity than children.
By the refutation of the second and third possibilities, the first assumption was strengthened in comparison.
Therefore, I had to conclude that the inverse solution for childrens' cortical activity failed to match the accuracy of the adults'.
Without the spatio-temporal distribution of condition effects in children, the task of determining their relevant TOI and ROI for the following section showed little promise.