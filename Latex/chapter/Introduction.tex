\chapter{Introduction}\label{intro}

\section{Functional anatomy of language processing}

\subsection{White fiber tracts}
Brain tries to save space and energy
Connections cost both space and energy, so brain is largely locally organized
Long-distance connections cost much more space and energy than local connections, while not transferring more information
The fact that long-distance connections exist means the transferred information is especially important and/or highly compressed

\subsection{Language network}
Dorsal \& Ventral streams
The ventral stream uses two major long-distance fiber tracts: ECFS and UF
Ventral pathway I (ECFS): STG to BA45
Ventral pathway II (UF): aSTG to FOP
The dorsal stream uses two major long-distance fiber tracts: The arcuate fascicle and the superior longitudinal fascicle.
Dorsal pathway I (SLF): pSTG to premotor cortex in BA6
Dorsal pathway II (AF): pSTG to BA44

\subsection{Cortical activity}
Region of interest: pSTS
pSTG is involved in complex syntax
"'Neuroimaging studies have tried to characterise the neu- ral substrate for representing human action. Many of these studies have followed a different tradition in psychophysics and developmental psychology of investigating the percep- tion of "'biological motion"' - that is, the characteristic articu- lated motion of chordate animal bodies (e.g. Vaina, Solomon, Chowdhury, Sinha, \& Belliveau, 2001; Grossman \& Blake, 2002; Beauchamp, Lee, Haxby, \& Martin, 2002; Pelphrey, Mitchell, McKeown, Goldstein, Allison, \& McCarthy, 2003) or body and face parts (e.g. Hoffman \& Haxby, 2000; Hooker et al. 2003; Kilts et al. 2003; Pelphrey et al., 2003). Biologi- cal motion can also be perceived from the relative motion of just a few dots ("'point-light walkers"', Johansson, 1973); if the dots are spatially or temporally rearranged, the percept is destroyed. These neuroimaging studies suggest that one brain region, the posterior superior temporal sulcus, is par- ticularly involved in the representation of biological motion.
Two sets of recent neuroimaging data suggest that the role of the posterior superior temporal sulcus (pSTS) may extend beyond a response to biological motion, to more ab- stract representations of intentional action. First, Castelli, Happe, Frith, \& Frith, (2000) and Schultz, Grelotti, Klin, Kleinman, Van der Gaag, Marois, \& Skudlarski, (2003) reported that a region of the pSTS showed a significantly higher response to animations of moving geometric shapes that depicted complex social interactions than to anima- tions depicting inanimate motion. Second, using movies of human actors engaged in structured goal-directed ac- tions (e.g. cleaning the kitchen), Zacks, Braver, Sheridan, Donaldson, Snyder, Ollinger, Buckner, \& Raichle, (2001) found that activity in the pSTS was enhanced when the agent switched from one action to another, suggesting that this region encodes the goal-structure of actions. Both of these results are consistent with a role for a region of pSTS cor- tex in representing intentional action, and not just biological motion."' (doi:10.1016/j.neuropsychologia.2004.04.015)

Accurate role in syntactic processing isn't clear yet
Utilizing dorsal pathway I and II for researching pSTG
Paradigm of choice: Social interaction, Object-/Subject-relative clauses
Previous findings: pSTG and IFG (BA44, 45, 47 + FOP) are more active in object-relative clauses


\section{Developmental aspects}

Kids don't have the AF tract yet
Previous findings: kids rely more on the ventral pathway
Ventral processing involves BA45 (no condition effect in adults)
Processing is more vulnerable to bias (semantic crosstalk)
Behavioral data is worse too


\section{Research questions}

Replication of EEG/fMRI results with MEG?
Which cortical regions are involved in the conditional effect?
Is semantic content relevant?
How long does each processing stage take?
In which order do the steps take place?
Can we see a different pathway in kids?

\subsection{Hypotheses}
Condition effect mainly in pSTG, BA44 (adults), BA45 (kids)
Kids: worse performance than adults
Kids: less involvement of pSTG
Adults vs kids: Dorsal II vs. Ventral II


\section{Choice of measurement methods}

\subsection{Acquisition}
MEG: Very high sampling rate
very little spatial distortion from the head
Low signal-to-noise ratio

\subsection{Forward models}
Based on individual high-resolution anatomical data
1-layer BEM: Robust creation, good software support
Quick and semi-automated process

\subsection{Localization}
Distributed source model: models a ton of unwanted activity
-> high SNR for actual ROI
sLORETA: no overlap effect at the boundaries, high ROI specificity

\subsection{Information transfer}
Meaning of information transfer
modelling vs. model-free, linear vs. nonlinear
Transfer Entropy is suited best for highly complex, nonlinear timescale data
Potential pitfall: Volume conduction (can also come from bad localization)
TrenTOOL can correct volume conduction
