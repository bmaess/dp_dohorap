
\chapter{Introduction}

\section{The role of white matter tracts in syntax processing}

\subsection{Exploration of white matter tracts}
The human brain can be regarded as an information-processing system, processing and storing environmental stimuli in order to create meaningful actions. 
In a very simplified information-processing perspective, it can be reduced to two functional components: computational units in the grey matter and signal relays in the white matter.
Contrary to other common information-theoretical systems, there is no fundamental distinction between hardware and software in the brain.
Both structural elements (e.g., dendritic spines) and computational units (e.g., astroglial cells) are part of an inseparable information process.
The fact that anatomical features can't be clearly mapped to a functional role violates the naive reductionist approach, and puts cognitive science more in line with other fields that explore dynamical systems, such as physics, theoretical computer science or meteorology.
Although the exploration of anatomic features (e.g., the classification of cortical regions based on myeloarchitecture) led to several ground-breaking discoveries in psychology, these findings tend to serve as a base for more specialized studies, rather than creating a framework for understanding cognitive processes.

Most anatomical features offered a direct link between their structure and functional roles, providing a solid basis for functional research.
In the context of the human brain, there was one major exception: cerebral white matter, comprised of myelinated axons and glial cells.
The role of cerebral nerve fibers eluded scientific inquiry for centuries, mainly due to their microscopic structure and the unknown function of their end connection points.
The idea that they played a role in signal transmission (in the loosest sense) can be traced back as early as 1710 \cite{1.1.history}.
In the writings of Pourfoir de Petit, nerve fiber bundles were theorized to transmit animal spirits from the cortex to the medulla oblongata.
The complete motor transmission chain as known today, including the path from from the medulla to the extremities via the spinal cord, was discovered only 40 years later by Emanuel Swedenborg \cite{1.1.history}.
However, during the next 250 years, the exploration of the role of cerebral fiber tracts was stifled by the lack of a non-invasive, large-scale measurement method.

These circumstances abruptly changed in 1972 \cite{1.1.history}, when the discovery of the autoradiographic method allowed tracing the precise interneural connections with radioactive markers.
Radioactive tracing was very widely used to fill existing knowledge gaps in cognitive science, but one fundamental flaw has prevented its application in humans: not only is the tracer very invasive, but the subject needs to be killed and dissected in the course of the experiment.
This drawback has prevented the study of the role of human fiber bundles, and not all knowledge gaps could be filled with equivalent results from lab animals.

One of the most important set of fiber bundles that eluded autoradiographic study was the human language network.
Only with the advent of main-stream diffusion imaging \cite{1.1.firstDiffusion} in the mid-1980s, examinations of human fiber bundles in vivo started to be possible.
The analysis of the previously elusive language-related fiber bundles is still ongoing, while new methods in both acquisition and data processing proceed to advance.
But before I give an overview over the language network, some fundamentals need to be explained first.

\paragraph{The role of myelination}
When regarding the brain from an information-theoretical perspective, it is important to consider that it is a self-optimizing structure with two important optimization constraints: the mechanism of myelination and limited space.

The importance of myelination becomes clear when regarding the mechanism of long-distance cerebral signal transmission.
Nerve fibers typically transmit signals in the form of electric fields between two sets of neuronal clusters.
The transmission process relies on the interplay of ion channels and electric currents.
Since the signal propagation is based on an activation cascade of ion channels, the transmitted signal is strictly binary in nature, is limited in bandwidth by a refractory period of $\approx 1ms$ and can only travel in one direction at a time.
The typical propagation speed of a nerve fiber is only $0.1\frac{m}{s}$, but can be increased by several magnitudes when enveloped by myelin layers.
Myelin layers, produced by myelinating sheath cells, reduce the amount of electrical current lost to the extracellular (usually liquid) environment.
This effect causes not only a higher propagation speed, but also an equally increased maximum signal bandwidth, since the refractory period stays constant.

\begin{figure}[h]
\begin{center}
\vspace{7mm}
\includegraphics[width=0.75\textwidth]{pics/1_1_myelin_sheath}
\caption{\label{1.1.neuron.illustrated} Illustration of an axonal fiber, wrapped in myelin sections. Adapted from Mosby's Medical Dictionary, 8th edition. \textcopyright 2009, Elsevier.}
\end{center}
\end{figure}

However, myelination has a major drawback: it increases the volume of the nerve fiber drastically.
For an practical illustration, A-$\gamma$-fibers, responsible for relaying hot and cold sensations, have a diameter of \SI{3}{\um} and propagate signals at $15\frac{m}{s}$.
In comparison, A-$\beta$-fibers, responsible for sensing touch on the skin, have a diameter of \SI{10}{\um} and propagate signals with a speed of $50\frac{m}{s}$.
The cross section of this fiber class is 11 times bigger, although its effective bandwidth is only three times higher.
The volume requirements of myelination follow a law of diminishing returns: a single fiber with a high propagation speed requires more space than two fibers at half the speed, although the combined bandwidth is equivalent.

Since the brain is situated in an inflexible skull cavity, overall volume for the whole network (and, in effect, for every nerve fiber) is fundamentally volume-constrained.
Myelination is flexible and adaptive, so bandwidth can be increased by adding additional myelin layers to fibers in high demand.
However, due to the volume constraints, this is only possible when neighboring fibers are reduced in diameter by the same volume.

This combination of constraints has apparently led to the prominent "'small-world"'' topology featured in every level of the central nervous system \cite{1.1.smallWorld}.
This type of network is characterized by a large number of very local connections; in fact, almost the majority (47.8\%) of all nodes are only connected to their nearest neighbors.
The small-world topology is a natural result of the trade-off that is created by minimizing the network distance between nodes while using the smallest amount of connections.
It is also responsible for one of the most convenient characteristics in cognitive science, the specialized brain region.
In a strongly local network, neural clusters in close proximity can share their information most effectively in closest network proximity.
And since neural clusters can choose to connect to other clusters most relevant to their task, this network proximity also becomes a functional proximity.
In a small-world-topology, this functional proximity becomes a spatial proximity.
Therefore, the optimal solution for information processing at any network level is to group neuronal clusters with similar tasks into a common spatial region.
As a result, the brain is highly structured into distinct functional regions, and regions in close proximity usually work together.

The combination of volume constraints and myelination properties creates another trade-off in one of the most fundamental tasks of the brain: learning.
Learning, on a neural level, always requires some form of change.
This change can occur within the neuron, for example by tweaking the input weights of established connections.
In other cases, new signal sources are necessary to accomplish the recently learned task.
Finding new signal sources requires movement, either by growing new fiber strands or by repositioning or remyelinating existing fibers.
In extreme cases, even the neuron body can move, at up to 10mm per hour.
Again, due to volume constraints, movement requires space, which is usually acquired by demyelinating existing fibers.
In order to preserve the effective bandwidth even in slower fibers, neural clusters at the end points have the option of compressing the signal.
An example for a compression strategy is to employ two clusters of redundant information at both ends of the fiber, and only transmit the reference to the cluster that contains this information.
Signal compression inevitably increases the local network complexity at the fiber end points, but can still create effective volume savings.
Especially in long-distance connections, fiber diameter has a very large impact on overall volume.
Long-distance connections make up only 2\% of all connections but occupy two thirds of white matter volume [1.1.fiberstats].

The volume constraints put a constant metaphoric pressure on both length and diameter of long-distance connections.
When a bundle of fibers maintains its length and myelination thickness beyond the first few years of development, it automatically implies that the underlying signals require both high bandwidth and high priority.

\subsection{Language network}
The language fiber network, as previously mentioned, has been explored only in the recent two centuries.
There are four fiber tracts that seem to be responsible for relaying language-related signals, which we can reasonably assume to be responsible for relaying signals with high bandwidth and priority.
All of those four fiber tracts connect regions of the frontal cortex with regions of the temporal cortex.
They can be grouped into two sets, a dorsal and a ventral set \cite{1.1.Gierhan}.

The dorsal set consists of the arcuate fascile (AF) and the superior longitudinal fascile (SLF).
The SLF consists of three individual sections, which connect different pairs of anatomical features.
SLF II connects the angular gyrus with the frontal cortex, SLF III connects the frontal cortex to the supramarginal gyrus and SLF-tp connects the angular gyrus with the temporal cortex.
The AF connects pars opercularis and the posterior superior temporal gyrus (pSTG).

The ventral set consists of the inferior fronto-occipital fascicle (IFOF, also known as ECFS) and the unicate fasicle (UF).
The IFOF connects the frontal cortex with the posterior temporal cortex, as well as to occipital and parietal regions.
The UF connects the anterior interior region of the frontal cortex and the anterior temporal cortex.

The regions at the end points of these fiber tracts are responsible for forwarding and receiving language-related signals.
In order to examine their role in language processing, it is necessary to define these regions with the least amount of functional overlap as possible.
For this purpose, it is helpful to include a definition based on functional roles, rather than purely by their anatomical properties.
This perspective \ref{1.1.pathways} yields much more narrowly defined functional regions, which are still directly connected by the four major pathways.
Because these connections are defined by their functional roles, Friederici refers to pathways rather than fiber tracts.
There are, as before, four pathways in two groups: two ventral pathways and two dorsal pathways.

\clearpage

\begin{figure}[h]
\begin{center}
\vspace{7mm}
\includegraphics[width=0.49\textwidth]{pics/1_1_tracts_ventral}
\includegraphics[width=0.49\textwidth]{pics/1_1_tracts_dorsal}
\caption{\label{1.1.tracts}Illustration of the most probable course of language-related fiber tracts, assembled in a meta-study \ref{1.1.Gierhan}. Left: ventral fiber tracts, right: dorsal fiber tracts. Numbers indicate Brodmann areas. Abbreviations: AG - angular gyrus, dPMC - dorsal premotor cortex, FOP - frontal operculum, Fpole - frontal pole, IFOF - inferior fronto-occipital fascicle, ILF - inferior longitudinal fascicle, MdLF - middle longitudinal fascicle, N. N. - nomen nescio, Occ - occipital cortex, Orb - orbitofrontal cortex, pSTG - posterior superior temporal gyrus, MTG - middle temporal gyrus, PTL - posterior temporal lobe, SLF II/III/-tp - second/third/temporoparietal component of superior longitudinal fascicle, SMG - supramarginal gyrus, Tpole - temporal pole, UF - uncinate fascicle, vPMC - ventral premotor cortex.}
\end{center}
\end{figure}

Ventral pathway I connects the temporal pole with the frontal operculum and the orbifrontal cortex via the UF.
Both frontal operculum and the anterior temporal cortex showed activity when taxed with sentence comprehension.
This activity persisted even in the absence of semantic content.
From this discovery, the current consensus was established that ventral pathway I is responsible for the parsing of simple syntax structures.

Ventral pathway II connects a series of wide-spread regions via the IFOF.
Starting at the occipital cortex, it connects to the pSTG/MTG region and terminates in three frontal areas: the frontal pole, frontal operculum and BA45.
Most areas connected to this pathway, including BA45, MTG, aSTG and pSTG, are known for their involvement in semantic language processing.
Based on the similarities, ventral pathway I was hypothesized to be responsible for semantic processing and comprehension.
This hypothesis was confirmed in a series of studies [1.1.tracts.lesions, 1.1.tracts.interference.a, 1.1.tracts.interference.b], which showed that semantic deficits appeared exclusively when physically interfering with the IFOF.

Dorsal pathway I connects Wernicke's area and the premotor cortex via the SLF.

Dorsal pathway II connects Broca's area and Wernicke's area via the AF.

% A recent model describing the neural language circuit with respect to the information flow [5] has suggested the particular roles of the two dorsal pathways to be as follows: the pathway connecting the posterior TC to the PMC, probably mediated via the PC, supports input driven auditory-to-motor mapping, as in speech repetition, whereas the pathway connecting Broca’s area and posterior STG directly (the AF) supports the processing of syntactically complex sentences, possibly by providing top-down predictions for the incoming input.

% - role of each cortical region
% - pSTG is known to be used in SRC/ORC grammar tasks (from fMRI)
% - information streams from and to pSTG -> indication of its role

% "During sentence processing, the IFG and posterior superior temporal regions have been proposed to exchange information via
%  the dorsally located Arcuate Fasciculus/Superior Longitudinal Fasciculus (AF/SLF; Friederici, 2011; Meyer, Obleser, Anwander, \& Friederici, 2012; Wilson et al., 2011). This proposal still awaits direct causal support, since it is based either on tractography in healthy participants (Catani, Jones, \& ffytche, 2005; Friederici, Bahlmann, Heim, Schubotz, \& Anwander, 2006; Glasser \& Rilling, 2008; Meyer et al., 2012; Parker et al., 2005; Saur et al., 2010; Weiller, Musso, Rijntjes, \& Saur, 2009) or data from sentence-processing-impaired conduction aphasics whose lesions mostly involve both gray- and white-matter damage (Baldo, Klostermann, \& Dronkers, 2008; Buchsbaum et al., 2011; Friedmann \& Gvion, 2003) and primary progressive aphasics (Galantucci et al., 2011; Wilson et al., 2010, 2011). The AF/SLF has been taken to be particularly relevant for the processing of sentences with complex word orders" (Friederici, 2011; Wilson et al., 2011).

\section{Functional anatomy of syntax processing}

Region of interest: pSTS
pSTG is involved in complex syntax

Adults:
"'Neuroimaging studies have tried to characterise the neural substrate for representing human action. Many of these studies have followed a different tradition in psychophysics and developmental psychology of investigating the perception of "'biological motion"' - that is, the characteristic articulated motion of chordate animal bodies (e.g. Vaina, Solomon, Chowdhury, Sinha, \& Belliveau, 2001; Grossman \& Blake, 2002; Beauchamp, Lee, Haxby, \& Martin, 2002; Pelphrey, Mitchell, McKeown, Goldstein, Allison, \& McCarthy, 2003) or body and face parts (e.g. Hoffman \& Haxby, 2000; Hooker et al. 2003; Kilts et al. 2003; Pelphrey et al., 2003). Biological motion can also be perceived from the relative motion of just a few dots ("'point-light walkers"', Johansson, 1973); if the dots are spatially or temporally rearranged, the percept is destroyed. These neuroimaging studies suggest that one brain region, the posterior superior temporal sulcus, is particularly involved in the representation of biological motion.
Two sets of recent neuroimaging data suggest that the role of the posterior superior temporal sulcus (pSTS) may extend beyond a response to biological motion, to more abstract representations of intentional action. First, Castelli, Happe, Frith, \& Frith, (2000) and Schultz, Grelotti, Klin, Kleinman, Van der Gaag, Marois, \& Skudlarski, (2003) reported that a region of the pSTS showed a significantly higher response to animations of moving geometric shapes that depicted complex social interactions than to animations depicting inanimate motion. Second, using movies of human actors engaged in structured goal-directed actions (e.g. cleaning the kitchen), Zacks, Braver, Sheridan, Donaldson, Snyder, Ollinger, Buckner, \& Raichle, (2001) found that activity in the pSTS was enhanced when the agent switched from one action to another, suggesting that this region encodes the goal-structure of actions. Both of these results are consistent with a role for a region of pSTS cortex in representing intentional action, and not just biological motion."' (doi:10.1016/j.neuropsychologia.2004.04.015)


"'hierarchy construction might be supported by left inferior frontal (particularly BA44/45) and posterior superior temporal regions: (Ben-Shachar et al., 2003, 2004; Caplan et al., 2002; Just et al., 1996; Röder et al., 2002; Stromswold et al., 1996)"'

"'the enhanced inferior frontal (BA 44/45) activation observed for these sentence types is engendered by highly specialized aspects of syntactic processing, namely by syntactic transformations (Ben-Shachar et al., 2003, 2004; Grodzinsky, 2000)."'

 "'More complex syntactic processing, 
 including the identification of hierarchical structures within a sentence, takes 
 place later at around 300ms to 500ms in the already mentioned pSTG/STS 
 (Cooke et al., 2002; Vandenberghe et al., 2002; Humphries et al., 2005; 
 Bornkessel and Schlesewsky, 2006; Kinno et al., 2008; Friederici et al., 2009; 
 Snijders et al., 2009; Santi and Grodzinsky, 2010; Newman et al., 2010). How-
 ever, this is not achieved by the pSTG/STS alone but, as Friederici argues, 
 critically involves the pars opercularis of the inferior frontal gyrus (IFGoper) in 
 Broca's  area  (Friederici et al., 2006; Makuuchi et al., 2009; Newman et al., 
 2010; Wilson et al., 2010)."' (Skeide 2013)

Children:

"'Children
 performed best on subject relatives, followed by object relatives, indirect
 object relatives, oblique relatives and, finally, genitive relatives. Diessel and
 Tomasello argued that children's superior performance on subject relatives
 reflected a processing effect whereby children prefer to pursue subject-
 extracted interpretations because they have a preference to build simple
 structures, which is based upon their considerable experience with simple
 nonembedded sentences. This is similar to early arguments made by Bever
 (1970, see also Bates \& MacWhinney, 1982; Slobin \& Bever, 1982; Townsend
 \& Bever, 2001), who argued that children use a canonical sentence schema
 (NVN) to interpret sentences"' (Kidd, Brandt, Lieven 2007)

Development of simple sentence processing is complete at age 3 (Friederici 2005)
Kids often don't have the AF tract yet until early adulthood
Previous findings: kids rely more on the ventral pathway
Ventral processing involves BA45 (no condition effect in adults)
Processing is more vulnerable to bias (semantic crosstalk)

\section{Experimental paradigm}

Task: Exploring role and timing of pSTG/pSTS activity in complex syntax processing
Actor/Undergoer setup
Object-/Subject-relative clauses

Sentences are comparable to Constable, 2004:
Subject/object-relative clauses
"'The biologist - who showed the video - studied the snake."'
"'The biologist - who the video showed - studied the snake."'
fMRI effect in:
Left: BA40, 44/45, 39, premotor cortex
Right: BA44/45, premotor cortex

Also comparable to Bornkessel, 2006:
specifically, cases E and F (unambiguous subject- and object-relative clauses with active verb)
significant fMRI effect in:
left: IFG (pars opercularis), pSTG, pSTS, inferior frontal junction, ventral premotor cortex, interparietal sulcus
right: interparietal sulcus

Repeated-measures design
Stationarity assumption

\section{Research questions}

\subsection{Expected discoveries}
Replication of EEG/fMRI results with MEG?
Which cortical regions are involved in the conditional effect?
Is semantic content relevant?
How long does each processing stage take?
In which order do the steps take place?
Can we see a different pathway in kids?

\subsection{Hypotheses}
Condition effect mainly in pSTG, BA44 (adults), BA45 (kids)
Kids: worse performance than adults
Kids: less involvement of pSTG
Adults vs kids: Dorsal II vs. Ventral II\section{Choice of measurement methods}

\subsection{Acquisition}

\paragraph{Neuroanatomic principles}
Neuronal activity creates a combination of electrical and magnetic fields.
On a cellular level, activating a neuron causes depolarization, which in turn creates a weak electric field.
Most neurons are equipped with a long axonal fibre that transmits a relatively strong postsynaptic signal.
Especially in pyramidal cells which are responsible for long-distance transmissions, this axon can span several centimeters in length.
Since signals along axons travel by a complex combination of transmitter binding, ion flux and electric fields, neuron-to-neuron data transmission exhibits three important restrictions.
First, a successful signal transmission requires a short refractory period until the next transmission is possible again.
This leads to the phenomenon that transmissions can only travel one way, and overlapping signals on the same fiber are impossible.
Second, axonal transmissions can only be binary.
If the minimum threshold voltage is reached, the signal will be transmitted at maximum speed along the entire fibre.
Below the threshold, no transmission can occur.
Third, maximum transmission speed is relatively low at approximately $100\frac{m}{s}$.
Other commonly used transmission fibers, for comparison, achieve speeds of $2\cdot10^8\frac{m}{s}$ (copper wire) or even $3\cdot10^8\frac{m}{s}$ (glass fibre).
Additionally, transmision speed depends on the thickness of the fiber, with the thinnest fibers transmitting as slowly as 0.1m/s.
This property causes a considerable delay when transmitting signals over macroscopic distances.
Signal delay in technical applications is generally small enough to be disregarded or considered an inescapable nuisance.
In the brain however, two interconnected neurons at opposite sides of the head can generate output simultaneously, but their signals will be offset by several miliseconds by the time they arrive.
For any neural network, delayed information is therefore an widely expected issue and must be properly factored into signal processing.

\paragraph {Available acquisition methods}
Due to the long axon and the low transmission speed, transmitted signals will create a electrical field that moves along the axon during a non-trivial time window.
Aggregated electric fields from thousands of similar signals can be acquired with an electroencephalograph (EEG).
This process involves measuring the voltage at two or more arbitrary points in the brain; typically, on the surface of the head or the cortex (which is called intracranial EEG, or iEEG).

Every electric signal that travels along a conducting wire also induces a magnetic field.
Since an axon is no exception to this rule, any transmitted neuron-to-neuron signal can also be acquired with a magnetic sensor.
A magnetoenecephalograph (MEG) consists of several dozens of these sensors distributed around the head.
MEG and EEG are currently the only non-invasive acquisition methods that measure brain activity with a milisecond resolution \cite{1.5.MEG.a}\cite{1.5.MEG.b}\cite{1.5.MEG.c}.
The focus of this project is on cognitive processes that require timespans in the single-digit second range to complete.
Considering these small time frames, a good temporal resolution is integral for yielding a sufficient amount of data from every processing step.
This is the reason why I decided to use a EEG/MEG method, while using the highest available temporal resolution.

\paragraph{Choice of acquisition methods}
Compared to EEG, the MEG-based measuring strategy has a few advantages and drawbacks.

The first difference between MEG and EEG is due to the sensor technology.
While both methods rely on strong amplification of very small input signals, only MEG uses superconducting sensors.
Contemporary superconductors require cooling with liquid helium.
Magnetic fields from neurons are also weaker than environmental magnetic noise by several magnitudes.
To limit measurements to neural activity, the MEG device need to be shielded with large quantities of highly magnetically permeable material (most commonly, an nickel-iron alloy).
These requirements makes MEG much less portable, and equally more expensive, than EEG.
Passive shielding already reduces environmental noise levels by 25-60db \cite{1.5.SNR}.
In addition to that, it is possible to dampen noise levels by another 60db with adaptive noise reduction \cite{1.5.SNR}.
Good adaptive noise reduction requires large external coils that can counteract outside magnetic fields.
The use of superconducting MEG sensors is necessary due to the extreme weakness of neural magnetic fields.
With the large amplification factors involved in both methods, amplifiers contribute a large amount of noise to the signal.
But since the amplifier noise depends directly on their operating temperature, suspending the amplifiers in liquid helium drastically lowers the noise level \cite{1.5.MEG.a}.
Generally, EEG and MEG are considered equally sensitive \cite{1.5.MEG.a}\cite{1.5.MEG.c}.
However, the addition of magnetic shielding can elevate signal-to-noise ratios in MEG measurements above equivalent acquisitions from EEG \cite{1.5.SNR}.

The second difference between MEG and EEG is the drastically different distortion from surrounding tissue.
For an EEG to be able to measure a potential difference on the skin surface, an electrical current needs to pass the tissues surrounding the cortical surface \cite{1.5.tissues.b}.
Some of these tissue layers, like blood vessels or cerebreal spinal fluid (CSF), are 5 times more conductive than gray matter; so they smooth and diffuse electric fields \cite{1.5.tissues.a}\cite{1.5.tissues.b}.
Other tissue layers, especially the compact bone, conduct electricity 78 times worse than gray matter; so they distort and attenuate every passing signal \cite{1.5.tissues.a}.
After these tissues have been passed, the original signal has substantially decreased in intensity, and changed drastically in shape and location.
In contrast to electrical fields, the same tissues are highly permeable to magnetic fields.
The magnetic permeability of water, which most human tissue is based on, differs from the magnetic permeability of vacuum only by 0.0008\%.
As a general rule, only metal-based materials are substantially less permeable than water.
Since the human head doesn't contain metal in considerable quantities, it practically allows magnetic fields to pass without distortion \cite{1.5.tissues.a}.

This circumstance does not imply, unfortunately, that every neural transmission arrives at the magnetic sensors with equal strength.
There are three main reasons for that.

First, the main source of electrical and magnetical activity are the pyramidal cells in the cortical tissue on the brain surface.
The activity of a single neuron, besides being highly unlikely in vivo, doesn't create a field with sufficient strength to elict a response in contemporary EEG or MEG sensors.
Only the combined and synchronous activity of larger clusters of neurons can cross the detection threshold.

Second, the human cortex is folded into gyri and sulci.
When two neural groups at opposite cortical walls produce identical activity, the two created electric and magnetic fields are directly opposed to each other.
Opposing fields cancel each other out, so the original signal will be systematically underestimated by surrounding sensors.

Third, their basic physical properties imply that electric and magnetic fields are orthagonal to each other.
Signals that travel along fibers orthogonally to the head surface create the strongest magnetic activation in surrounding sensors, but the weakest voltage in surface electrodes.
Fibers that lie parallel to the head surface, in contrast, create strong electric but weak magnetic activation.
The implication on measurements of neural activity is that EEG is most sensitive for gyri and sulci, and MEG is most sensitive for the radial walls inbetween.
To counteract this particular measurement bias, EEG and MEG data would have to be acquired simultaneously.

Although the simultaneous acqusisition of EEG and MEG data provides theoretical benefits to the signal quality, I ultimately decided against this strategy.
MEG acquisition consists of three preparation steps: Getting written consent, applying the HPI coils and ocular electrodes, and digitizing the head.
This preparation typically requires 25 minutes for children and 15 minutes for experienced adults.
Including EEG acquisition would have added several lengthy steps to this procedure: Fitting the gel electrode cap, ensuring a good connection for each electrode, and plugging in each of the 63 cables individually.
These additional steps would have extended the preparation time to at least 60 minutes.
Since patience is not a strong trait in children\footnote{Implementing details with the goal to prevent boredom was a common theme in this study, as explained in more detail in chapter 3.}, I wanted to minimize any idle waiting times between their arrival and the experiment.
Therefore, I only acquired MEG data during this study.

\subsection{Preprocessing}

Once the signals were acquired by the MEG, there were two necessary decisions for preprocessing.

\paragraph{DC bias removal}
The first decision concerns the removal of DC bias that is often caused by slow sensor drift.
Traditionally, for the exploration of effects in ERP and ERF, a subtractive baseline correction is applied to every trial before averaging.
For this purpose, the average is computed from roughly 100 to 500 (typically 200) miliseconds of activity before the conditional cue.
This initial interval is assumed to originate from brain activity unrelated to the post-cue task.
The computed average value is then subtracted from activity data in the corresponding trial.
This procedure assures that different DC components from long-term trends (for either technical or cognitive reasons) don't disturb the trigger-dependent effect.
However, there is a fundamental issue with the baseline correction.
Because the stimuli are spoken sentences in this study, there is no silent time window around the critical words.
Therefore, the pre-cue interval reflects electric fields from unrelated brain activity.
By computing the average from unrelated activity, I effectively introduces a random DC error into the correction procedure.
This issue makes a subtractive baseline correction a tradeoff between the original DC error and a random DC error.
A process to remove DC components without this tradeoff is to use a highpass filter \cite{1.5.highpass}.
Since my longest expected evoked field, the ELAN, has a base frequency of 5Hz, I choose a much lower value of 0.4Hz for the high-pass.

\paragraph{Artifact removal}
The second decision concerns the removal of various measuring artifacts.
There are four major types of artifacts during the acquisition of electric or magnetic fields.

The first type of artifact is caused from cardiac activity.
Cardiac muscles create an almost continuous, very regular field with low frequency and medium strength.

The second type of artifact is caused by ocular movements.
Since the eyeballs are electrically charged, all eye movements are associated with a continuously changing field of low frequency and medium strength.

The third type of artifact is caused by muscle movements.
Muscle activity creates relatively long and strong distortions in a wide frequency band.

The final type of artifact is caused by oversaturation in MEG sensors.
Oversaturation happens randomly when no high pass is in use, and reduces the sensitivity of the affected channel to zero.
This condition is remedied with an automatic reset, which in turn produces a single very short and very large jump in amplitude.

The first two artifacts can be eliminated with the help of three additional acquisition channels.
With electrodes attached to the chest and to the eye sockets, electric fields from ocular and cardiac activity are measured directly.
MEG data is then deconstructed into independent data components with an independent component analysis.
The artifact channels are used to identify artifact components in the measured MEG data.
If the extracted data component is similar enough to one of the measured artifact channels, it is removed. 
The remaining components are then assembled to a data composition, ideally containing no cardiac or ocular artifacts.

The last two artifacts can be removed with a simple threshold detection.
Their high amplitude make it possible to set a manual amplitude threshold, and reject segments that exceed this threshold in any channel.
I determined the threshold manually after visual artifact inspection, and decided to reject entire trials if this threshold is exceeded.

\subsection{Timewindow estimation}
The acquired signals need to be explored for the impact of the conditional effect.
This effect is usually spatially and temporally limited.
For establishing time intervals (TOI) and spatial regions of interest (ROI), there are two possible approaches.

First, existing literature can be consulted for activity effects from syntax contrasts in similar experiments.

Second, a bootstrapping approach can be used.
For this approach, the measured activity is compared between conditions.
The TOI and ROI that involve considerable contrast between conditions can then be selected for the comparison of mean activity.
The drawback to this approach is both spurious contrast and activity from different cognitive processes are considered as condition effect.
The statistical testing will therefore systematically overestimate the condition effect.
This issue is known as "'double dipping"' \cite{1.5.Kriegeskorte}.
However, for exploratory analysis, bootstrapping is a valuable tool that can uncover previously unknown TOI and ROI.

I decided to use both approaches with different purposes:
First, comparisons within previously discovered ROI and TOI provide results that can be compared well to the findings of earlier studies.
Second, a bootstrapped comparison allows for the exploration of spatial and temporal properties of the syntactic effect.

\subsection{Source localization}

\paragraph{Motivation}
Magnetic fields, when induced by the brain, arrive at MEG sensors only as mixture of many cortical sources.
Demixing these signals is a fundamentally flawed process.
The problem of discerning these signal sources is equivalent to finding the location and intensity of all the flames in a hot air balloon, while only looking at the outside of the hull.
There are infinitely many possible configurations of light sources that can generate the same brightness pattern on the outer skin.
The same issue is valid for localizing magnetic signal sources in the human head.
This task involves creating a bidirectional map between the curved plane of MEG sensors and the threedimensional human head.
Because of the different dimensionality, the task of creating this map is an underdefined problem.
This means that there are infinitely many possible locations and intensities for magnetic fields that can generate the exact same signal pattern in the MEG sensors.
This multitude of possible solutions needs to be constrained to make the results meaningful.
One popular set of constraints is the use of a source model.

\paragraph{Choice of source model}
A source model assumes that there is a limited number of discrete current sources distributed throughout the brain.
Usually, these current sources are assumed to be generated by neuronal tissue.
There are three popular types of source modelling.

The first type of source modelling uses spatial filtering.
The most common spatial filtering strategy is the single-core beamformer method \cite{1.5.Beamformer-a, 1.5.Beamformer-b}.
Its main weakness is the assumption that data from different sources is completely uncorrelated.
This assumption is especially detrimental to the analysis of cortical signals, since neuronal-level synchronizity is one of the fundamental principles behind attention and learning \cite{1.5.synchronizity}.

In the second type of source modelling, the focal source model, neural current flow is represented by a limited set of point-shaped current dipoles.
There are are three subcategories to this model: an unconstrained variant (the moving dipole model), dipoles with a fixed position (the rotating dipole model) and dipoles with a fixed position and rotation (the fixed dipole model).
Popular applications of this approach include "'multiple signal classification"' (MUSIC) \cite{1.5.music} and "'multi-start spatio-temporal multiple-dipole modeling"' \cite{1.5.simplex}
Dipole models have had limited success with representing neuronal responses for two main reasons.
First, reducing extended neuroanatomical structures to a point current source introduces a systematic model error.
Second, the number and location of dipoles has a strong influence on the localized results, yet is hard to estimate in advance (Huang et al., 1998).

The third type of source model is the distributed source model.
For these approaches, a dense grid of dipoles is derived from a cortical layer.
The goal is to place dipoles in homogenous density at every location that is able to produce currents.
Typically, the continuous cortex surface is extracted from anatomical data and populated with several thousands of (roughly) equally spaced dipoles.
For determining localized activity, moments are computed for all dipoles.
The dipole moments are then used to simulate activity in the MEG sensors.
Simulated activity then is optimized so that the error to the reference sensor activity is minimal.
Because every sensor activity pattern can be created by infinitely many source configurations, the localization process is facilitated with two processes.
First, dipole activity is spatially regularized with a predefined factor.
Second, the best pattern of localized sources is selected by minimizing the norm over all dipoles.
The most popular norm today is the L2-norm \cite{1.5.L2}, and implementations are widely available.
Alternatively, the L1-norm \cite{1.5.L1} can result in a more focally reconstructed activity.
This process is usually computed separately for every temporal sample.
Popular implementations include dSPM \cite{1.5.dSPM}, MNE \cite{1.5.MNE} and sLORETA \cite{1.5.sLORETA}).
I opted for this approach because the spatial filtering approaches aren't recommended for localizing cortical activity, and the quality of results from the dipole fit models depend too strongly on the inital parameters.

L2-norm-based solutions have two major drawbacks.
First, the solution has a relatively low spatial resolution.
This issue leads to spatially distributed activity clusters even if the real sources are very focal.
If the sources are in close proximity, unintended mixing of reconstructed source activity can occur as well.
Second, generic L2-normal solutions contain a mandatory systematical spatial bias.
The sLORETA algorithm, by contrast, has been designed to create solutions with zero bias.
Since this algorithm has minimal drawbacks out of all readiliy available software, it became the method of choice for my source localization purposes.

\subsection{Information transfer}
Cognitive processes are based on the interaction of dynamically and statically coupled neural networks.
Statical coupling is established with nerve fiber connections, mainly representing the upper limit of possible functional connections to other areas.
The other type of coupling are dynamic connections, which establish the exchange of information by synchronizing neuron clusters to a common frequency.
Especially the latter coupling strategy plays a big role in understanding cognitive processes, since dynamic connections are - due to bandwidth constraints - only upheld as long as meaningul information needs to be transfered.
By measuring the time intervals and spatial extends of dynamic connections, it is possible to construct a spatiotemporal graph of interactions between all cortical regions involved in a cognitive task.
My task was to determine two maps of dynamic connections, one for each experimental condition, and compare the differences.

\paragraph{Two types of information interaction}
According to the Wiener principle \cite{1.5.information}, information processing in general consists of three separate tasks: information storage, information modification and information transfer.
This definition is important for determining the task that can be measured with practical methods.
In the literature, a common question is whether a signal transfer was causal or not.
However, reconstructing true causal relationships between signal processors involves measuring all three tasks.
Both information storage and information modification take place on a cellular level.
Since these processes leave few traces in the form of electric or magnetic fields, our only way to resolve their details is by analyzing single cells with microscopic probes.
In contrast to these tasks, information transfer can reconstructed reasonably well from discrete, brain-wide electrophysiological measurements.
Although the practical restrictions limit us to the analysis of information transfer, this measure, rather than true causality, may actually be the more interesting factor for understanding a computational process in the brain [1.5.causality].

Historically, information transfer has been explored by calculating functional connectivity \cite{1.5.connectivity}.
Functional connectivity is defined as the temporal correlation between activity of different cortical regions.
While this approach is suitable for a wide spectrum of acquisition methods (EEG, MEG, fMRI, PET), it fails to uncover the directionality of any interaction.
For this goal, a relatively new approach is necessary, effective connectivity.
Next to functional and structural (anatomical) connectivity, this is a fundamentally different type of approach.
It explores the influence that one cortical region has over another cortical region, and is uniquely able to determine directional properties of signal interactions.

\paragraph{Choice of analysis approach}
There are two fundamentally different approaches to the analysis of effective connectivity.

The first type of analysis involves fitting each acquired time series to a well-defined model.
These types of models approaches commonly involve on biophysical properties, e.g. varying neuronal firing rates or varying strengths of dendritic connections.
Once the model parameters are established, the implied interactions between the models can be explored.
The commonly used model-based approaches are statistic equation modelling \cite{1.5.SEM} and dynamic causal modelling \cite{1.5.DCM}.
Due to their large parameter space and high computational demands, these approaches are better suited for small networks.
I ultimately decided against a model-based approach because of the non-trivial task of parameter optimization.

The second type of analysis works without an underlying model or other metadata.

\paragraph{Granger-Causality derivatives}
This data-driven approach was incepted with Granger causality \cite{1.5.Granger}, which is built on Wiener's assumption that causes precede their effects in time.
If the future time series of a signal can be better predicted by the past time series of a second signal than by its own past, then the second signal has had a causal effect on the first signal.
There are different implementations within this family of approaches with minor conceptual differences.
Granger causality, for example, can accomodate a maximum of two time series.
An effort to generalize this method to multiple data sources and to frequency space led to the development of the directed transfer function.

Importantly, all following methods are robust against a common effect in realistic electrophysical data: volume conduction \cite{1.5.PDC}.
Since the brain is enveloped in cerebral spinal fluid, a highly conductive material, any cortical activity can spread practically immediately across wide distances.
Due to its small amplitude and almost random content, this effect is negligible for traditional trigger-locked approaches, in which a signal stationarity can be assumed.
However, during the analysis of effective connectivity, volume conduction leads to an instantaneous signal transfer, competing with the much more meaningful signal transfer by axonal fibers.
Volume conduction especially overestimates the amount of signal transfer in EEG data, in which the electrical activity from neuronal clusters are mixed with electric fields in the liquid layer around the cortex.

Because the directed transfer function fails to distinguish between direct and indirect interactions in a multivariate network, the partial directed coherence (PDC) method was developed.
PDC, which represents data as a multivariate autoregressive model, is able to select only direct interactions.
PDC shows a few minor issues, most famously its susceptibility to different levels of amplitude.
All of these issues are resolved with the generalized PDC (gPDC) \cite{1.5.gPDC}.

All derivative approaches of Granger causality can only capture linear interactions between data streams.
This limitation leads to a systematic underestimation of effective connectivity when non-linear interactions are involved in creating the measured activity.

\paragraph{Information-theoretic approaches}
Next to mutual information and joint entropy, transfer entropy (TE) is by far the most advanced information-theoretic approach \cite{1.5.TEcomparison}.
TE is a non-linear, model-free approach, is resistant to indirect causal effects and volume conduction, and incorporates both dynamical and directional information \cite{3.4.TE}.
It is, just like Granger causality, based on Wiener's assumption of temporal delays between cause and effect.
The core principle is the comparison within and between two streams of entropy.
If the future entropy of one data stream can be explained better by adding entropy from another data stream, an entropy transfer must have occurred.
This entropy transfer is quantified and secured against false positive detection with a statistical comparison against surrogate data.
This surrogate data needs to be constructed so that the causal relationship between data streams is destroyed.
TE also requires relatively large amounts of data, which shows at least some stationarity.
Both requirements are solved by supplying trigger-locked single trials with generous time limits and a high sampling rate.
The most extensive implementation of TE, TrenTOOL, allows for computing group-level statistics and impacts of conditional effects, making it the ideal software for the goal of my study.